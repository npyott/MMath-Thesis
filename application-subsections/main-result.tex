We are now ready to go over our main result from \cite{Silverman_2004}.

\begin{theorem}
    Let $X/k$ be smooth with $Y \subseteq X$ a smooth subvariety of co-dimension $r \geq 2$.
    Let $A$ be some ample divisor on $X$,
    and assume that $-K_X$ is a normal crossings anti canonical divisor such that $\supp(-K_X) \cap Y = \emptyset$.

    Assuming Vojta's conjecture,
    then for every finite set $S \subseteq M_k$ and any $0 < \varepsilon < r - 1$,
    there is a closed subvariety $Z \subsetneq X$ and a constant $\delta \in \R$,
    with $\delta$ only depending only on $X$, $Y$, and $A$,
    such that for any $P \in (X \sm Z)(k)$
    \[
        h_{\gcd}(P; Y)
            \leq \varepsilon h_{A}(P) + \frac{1}{r - 1 + \delta \varepsilon} N_S(-K_X, P) + O(1).
    \]
\end{theorem}

\begin{proof}
    Let $\pi : \wt X \to X$ be the blow up of $X$ with centre $Y$ and exceptional divisor $\pi^{-1}(Y) = E$.
    We begin by recalling that the canonical bundle $K_{\wt X}$ may be described with the given codimension as
    \[
        K_{\wt X} = \pi^* K_X + (r - 1)E,
    \]
    up to linear equivalence.
    Next, since $A$ is an ample divisor,
    we may find some $m \geq 1$ and $\wt A \in \Pic(\wt X)$ that is ample such that
    \[
        \pi^*A = \wt A + m E.
    \]

    As $\supp(K_X) \cap Y = \emptyset$,
    it is clear as well that $-\pi^* K_X$ is a normal crossings divisor as $\pi$ is an isomorphism at relevant points.
    Thus, we are ready to apply Vojta's conjecture to $-\pi^*K_X$ and $\wt A$.

    Taking $\varepsilon > 0$ and $S \subseteq M_k$ as above,
    we may assume that away from some exceptional proper closed subset $W$,
    for any $\wt P \in \wt X \sm W$,
    \begin{align*}
        & m_S(-\pi^*K_X, \wt P)
        + h_{K_{\wt X}}(\wt P)
        \leq \varepsilon h_{\wt A}(\wt P)
        + O(1) \\
        \Rightarrow
        & -m_S(K_X, \pi(\wt P))
        + h_{K_X}(\pi(\wt P))
        + (r - 1)h_{E}(\wt P)
        \leq \varepsilon h_{A}(\pi(\wt P))
            - \varepsilon m h_{E}(\wt P)
            + O(1)
    \end{align*}
    Next, we may as well assume that $E \subseteq W$,
    and so there is some $P \in (X \sm Y)(k)$ such that $\pi(P) = \wt P$.
    Recalling our definition of the generalized GCD,
    and simplifying using the counting function $N_S(-K_X, P)$,
    we may then write
    \[
        N_S(K_X, P)
        + (r - 1)h_{\gcd}(P; Y)
        \leq \varepsilon h_A(P)
        - \varepsilon m
        h_{\gcd}(P; Y) + O(1).
    \]
    After just one more re-arrangement by grouping terms,
    and setting $\delta = m$,
    we finally arrive at
    \[
        h_{\gcd}(P; Y)
        \leq \frac{\varepsilon}{r - 1 + \delta \varepsilon} h_A(P)
        + \frac{1}{r - 1 + \delta \varepsilon} N_S(K_X, P)+ O(1).
    \]
\end{proof}

Notice that the above theorem is indeed a general case of the special result on Vojta's conjecture mentioned in a previous section.
Indeed, let $C/k$ be an elliptic curve with identity $O \in C$,
and choose any ample $A$ divisor on $X = C \times C$.
Notice that since $C$ is an elliptic curve that $K_X = 0$,
satisfying our necessary conditions and simplifying our inequality.
Moreover, if $K$ is the canonical divisor on $\pi: \wt X \to X$ blown up at $Y$,
then $K$ is linearly equivalent to the exceptional divisor by our co-dimension and that the canonical sheaf is trivial on $X$.
Thus, we may write for $P \in (X \sm Y)(k)$
\[
    h_K(\pi^{-1}(P)) 
    = h_{\gcd}(P; Y) 
    \leq \varepsilon h_A(P) + O(1)
    = \varepsilon h_{\pi^* A}(\pi^{-1}(P)) + O(1).
\]