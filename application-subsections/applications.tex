% Non-S component
\begin{definition}
    Let $S$ be a finite set of rational primes. For any integer $x \in \Z \sm \{ 0 \}$, The ``prime-to $S$" part of $x$, denoted $|x|_S'$, is the unique multiplicative component of $x$ which does not lie over the primes of $S$. That is,
    \[
        |x|_S' = |x| \cdot \prod_{p \in S} |x|_p = \left( \prod_{p \notin S} |x|_p \right)^{-1}
    \]
\end{definition}

% First Generalization
\begin{theorem}
    Let $V \subset \bbP^n$ be a smooth variety of co-dimension $r = n - \dim(V)$ not intersecting any of the hyperplanes $\{x_i = 0\}$ for $0 \leq i \leq n$.
    Suppose as well that $V$ is given as the vanishing set of some homogeneous polynomials $f_1, \ldots, f_m \in \Z[x_0, \ldots, x_n]$.
    We will also fix some $0 < \varepsilon$ arbitrarily.

    Suppose that Vojta's conjecture is true in the case of $\bbP^n$ blown up along $V$.
    Then we may determine some non-zero homogeneous polynomial $g \in \Z[x_0, \ldots, x_n]$ depending on the polynomials defining $V$ and $\varepsilon$, as well as a constant $\delta$ only depending on $f_1, \ldots, f_m$, so that every coprime integer tuple $(a_0, \ldots, a_n) \in \Z^{n + 1}$ is either a root of $g$ or
    \[
        \gcd(f_1(a_0, \ldots, a_n), \ldots, f_k(a_0, \ldots, a_n))
            \leq \max\{|a_0|, \ldots, |a_n|\}^\varepsilon
                \cdot (|a_0 \cdots a_n|_S')^{\frac{1}{r - 1 + \delta \varepsilon}}.
    \]
    
\end{theorem}

\begin{proof}
    Let $S \subseteq M_\Q^0$ be a finite set of rational primes and $\varepsilon > 0$ taken arbitrarily. 
    Note that showing it for $\varepsilon < r - 1$ suffices to prove for $\varepsilon \geq r - 1$, 
    so we also assume this without loss of generality.
    Using our main result with $X = \bbP^n$, $Y = V$, and $A = \{x_0 = 0\}$,
    we may find some proper closed subset $Z \subsetneq \bbP^n$ and $\delta$ depending only on $V$ (assuming $\bbP^n$ and $\{ x_0 = 0 \}$ fixed),
    for which any $P \in (X \sm Z)(k)$ satisfies
    \[
        h_{\gcd}(P; V)
        \leq \varepsilon h_A(P)
            + \frac{1}{r - 1 + \delta \varepsilon} N_{S \cup \{\infty\}}(-K_{\bbP^n}, P) + O(1).
    \]
    Note we may take some homogeneous $g' \in \Z[x_0, \ldots, x_n]$ for which $D_+(g) \subseteq X \sm Z$,
    we may assume that $Z$ is contained in the vanishing of $g'$.
    Additionally, as $V$ does not intersect any hyperplane of the form $\{x_i = 0\}$ for $0 \leq i \leq n$,
    we will assume $g = x_0 \cdots x_n g'$.

    To begin, we must consider local height functions for the anti-canonical divisor class $-K_{\bbP^n} \cong \cO(n + 1)$.
    To do so, for each $0 \leq j \leq m$ we may define local height functions for the divisor $\{x_j = 0\}$ for each $v \in M_k$ and $P \in (\bbP^n \sm \{x_j = 0\})(\Q)$ as
    \[
        \lambda_{\{x_j = 0\}, v}(P)
        = \log \max(|\tfrac{x_0}{x_j}|_v, \ldots, |\tfrac{x_n}{x_0}|_v, 1).
    \]
    By additivity, we may take $\lambda_{-K_{\bbP^n}, v}(P) = \lambda_{\{x_j = 0\}, v}(P) + \cdots + \lambda_{\{x_n = 0\}, v}(P)$ away from the set $\{x_0 \cdots x_n = 0 \}$,
    ignoring any $O_v(1)$ terms by this choice of local height function.
    Note this divisor representative for the anti canonical class is a normal crossings divisor.
    
    If we are given that $P = (a_0 : \cdots : a_n) \in \bbP^n(\Q)$ not vanishing on $x_0 \cdots x_n$,
    assuming without loss of generality that $\gcd(a_0, \ldots, a_n) = 1$,
    we obtain
    \begin{align*}
        N_{S \cup \{\infty\}}(-K_{\bbP^n}, P)
        & = \sum_{p \notin S}  \sum_{j = 0}^n \log \max(|\tfrac{a_0}{a_j}|_p, \ldots, |\tfrac{a_n}{a_j}|_p, 1) \\
        & = \sum_{p \notin S} \sum_{j = 0}^n \big(
            \log \max(|a_0|_p, \ldots, |a_n|_p) - \log |a_j|_p
        \big) \\
        & = - \log \prod_{p \notin S} |a_0 \cdots a_n|_p \\
        & = \log |a_0 \cdots a_n|_S'.
    \end{align*}

    Next, we will take our ample divisor $A$ to be given as the hyperplane $\{x_0 = 0\}$. 
    As the map $\bbP^n \to \bbP^n$ given by $A$ may be taken as the identity,
    we may simply write the formula for projective heights with co-prime integer coordinates as
    \[
        h_A(a_0 : \cdots : a_n) = \log \max(|a_0|, \ldots, |a_n|). 
    \]

    Finally, with our result on $h_{\gcd}(P; V)$ as shown in a previous section,
    we have deduced that for any $(a_0, \ldots, a_n) \in \Z^{n + 1}$ coprime and not vanishing on $g$,
    \begin{multline*}
        \log \gcd(f_1(a_0, \ldots, a_n), \ldots, f_m(a_0, \ldots, a_n)) \\
        \leq \varepsilon \log \max (|a_0|, \ldots, |a_n|)
            + \frac{1}{r - 1 + \delta \varepsilon} \log |a_0 \cdots a_n|_S' + O(1),
    \end{multline*}
    which is a logarithmic version of our desired inequality.
\end{proof}

\begin{remark}
    It should be noted that this theorem gives a conditional proof of \cite{BCZ_2002}.
    Indeed, fix integers $a, b \in \Z$ multiplicatively independent and let $\varepsilon > 0$ be arbitrary.
    Using $f(x, y, z) = x - z$, $g(x, y, z) = y - z$, and $S = \{ p \in M_k^0 : p | ab \}$,
    we obtain some polynomial $h(x, y, z)$ such that for all $n \geq 1$,
    either $h(a^n, b^n, 1) = 0$, or
    \[
        \gcd(a^n - 1, b^n - 1) \leq \max\{|a|, |b|\}^{\varepsilon n}.
    \]
    Since we have already shown that a polynomial $h(x, y, z)$ may vanish at only finitely many triples $(a^n, b^n, 1)$ without being trivial,
    the result follows.
\end{remark}

\begin{theorem}
    Let $E/ \Q$ be an elliptic curve given by a Weierstrass equation.
    Assuming Vojta's conjecture for $E^2$ blown up at $(O, O)$,
    we then find that for any $\varepsilon > 0$ that there is a proper closed subvariety $Z \subseteq E^2$ such that for any points $P, Q \in E(\Q) \sm Z$,
    \[
        \gcd(D_P, D_Q) \leq (H(P) \cdot H(Q))^\varepsilon,
    \]
\end{theorem}

\begin{proof}
    As before, we begin by using our main result.
    Set $X = E \times E$, $A = p_1^* O + p_2^*O$, $Y = (O, O)$, and $S = M_\Q^\infty$,
    where $p_1, p_2 : E \times E \to E$ are the projection maps.
    Since $K_E = 0$ for an elliptic curve,
    it follows that $K_X = 0$ as well.
    Therefore, for any $\varepsilon > 0$,
    there exists some closed subset $Z \subsetneq X$ such that for all $(P,Q) \in (X \sm Z)(\Q)$,
    \[
        h_{\gcd}((P, Q); (O, O))
        \leq \varepsilon h_{A}(P, Q) + O(1).
    \]

    To make a quick understanding this, we start on the left as
    \[
        h_{\gcd}((P, Q); (O, O)) = \log \gcd(D_P, D_Q).
    \]
    On the right, we may use additivity and linearity to find
    \[
        h_{A}(P, Q)
        = h_{O}(P) + h_{O}(Q) + O(1).
    \]
    Putting these results together, and rephrasing the result exponentially, 
    we obtain our desired result.
\end{proof}