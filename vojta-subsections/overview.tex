At the heart of this paper's conclusion lies a conjecture of Vojta.
To explore this conjecture further, 
let's re-examine the Subspace Theorem.

Fix a number field $k$, with algebraic closure $\ol{k}$ and ring of integers $R_k$,
and let $S \subseteq M_k$ be a finite set of places containing all the archimedean places.
Let's also consider a set of linearly independent linear forms $L_0, \ldots, L_n \in \ol{k}[x_0, \ldots, x_n]$.
Taking $\delta > 0$, our inequality of the Subspace Theorem is
\[
    \prod_{i = 0}^{n} \prod_{v \in S} \|L_n(x_0, \ldots, x_n)\|_v
    \geq (\max_{0 \leq i \leq n, v \in M_k^\infty} \|x_i\|_v)^{-\delta},
\]
for all $x_0, \ldots, x_k \in R_k$ except for those solutions contained in finitely many hyperplanes.
However, notice that through scaling both sides of the inequality by the same constant, 
this inequality may be viewed for points of $\bbP^n(k)$ except for an exceptional union of hyperplanes denoted $Z$.

To proceed, we note by Lemma 2.2.2 of \cite{Vojta_2006} that 
\[
    \max_{0 \leq i \leq j, v \in M_K^\infty} \|x_i\|_v
    \ll H_k(x)
\]
So we may generally write that outside of $Z$ we have for $\varepsilon > 0$ sufficiently small that
\[
    \prod_{i = 0}^{n} \prod_{v \in S} \|L_n(x)\|_v
    \geq H(x)^{-\varepsilon}.
\]
Notice as well that this reformulation is independent of our base field $k$,
so long as it is still a finite extension of $\Q$.

Next, consider the divisor $D = \{ L_0 \cdots L_n = 0 \}$.
We know for each $v \in M_k$ that our corresponding local height function can be written out as
\[
    \lambda_{D,v}(x)
    = \log \max_{1 \leq j \leq n}\left\|
        \frac{x_j^{n + 1}}{L_0(x) \cdots L_n(x)}
    \right\|_v
    = \log \max_{1 \leq j \leq n} \prod_{0 \leq i \leq n} \left\|
        \frac{x_j}{L_i(x)}
    \right\|_v.
\]

Rearranging our main inequality and taking logarithms,
we may now write that
\[
    \sum_{v \in S} \lambda_{D,s}(x)
    \leq \varepsilon h(x) + (n + 1)\sum_{v \in S} \log \max_{0 \leq j \leq n} \|x_j\|_v.
\]
To simplify this, note that $\log \max_{0 \leq j \leq n} \|x_j\|_v \geq O(1)$ for any $v \in M_k^0$.
To see why, 
recall that the class group of $R_k$ is finite,
and so we may fix a subset of small primes to which we scale our coordinates to having potentially positive division at.
Therefore, we may replace the sum over the places in $S$ with the full sum over all of $M_k$ while preserving inequality.

Lastly, note that on $\bbP^n$,
our canonical divisor $K_{\bbP^n}$ is given by the invertible sheaf ${\cO(-n - 1)}$, 
and so we may replace $-(n + 1)h(x) = h_{K_{\bbP^n}}(x) + O(1)$ to obtain
\[
    \sum_{v \in S} \lambda_{D,s}(x)
    + h_{K_{\bbP^n}}(x) \leq \varepsilon h(x) + O(1).
\]

With the above, 
we have found that the Subspace Theorem is actually a geometrical result on the space $\bbP^n$.
To explore this relationship in more generality,
let's go over some terminology.

\begin{definition}[Proximity Function and Counting Function]
    Given a divisor $D$ and a finite set of places $S \subseteq M_k$ for a number field $k$,
    we denote the \textit{proximity function} $m_S(D, P)$ for each $P \notin \supp(D)$ as
    \[
        m_S(D, P) = \sum_{v \in S} \lambda_{D,s}(x).
    \]
    Intuitively, this is minus the logarithm of the distance from $P$ to $D$ on the places $S$ of interest.

    We similarly define the complimentary \textit{counting function} $N_S(D, P)$ for each $P \notin \supp(D)$ as
    \[
        N_S(D, P)
        = \sum_{v \notin S} \lambda_{D, s}(x)
        = h_D(P) - m_S(D, P)
    \]
\end{definition}

%https://stacks.math.columbia.edu/tag/0CBN
\begin{definition}[Normal Crossings Divisor]
        A \textit{strict normal crossings divisor} $D$ on a non-singular scheme $X$ is an effective Cartier divisor $D$,
        such that for any $P \in D$ as a closed subscheme,
        there is a $\kappa(P)$ basis for $\Fm_P/\Fm_P^2$ with representatives $f_1, \ldots, f_d \in \Fm_P \sm \Fm_P^2$ for which $D$ is cut out by $f_1, \ldots f_r$ in $\cO_{X,P}$ for some $1 \leq r \leq d$.
\end{definition}

\begin{remark}
    Normal crossings divisors generalize the concept of hyperplanes in general position.
    Indeed, for a projective non-singular variety $X$,
    we may take a choice of coordinates of the surrounding projective space for which each hyperplane is some coordinate vanishing,
    and then these coordinates also generate any particular maximal ideal $\Fm_P$ for $P \in X$.
\end{remark}

\begin{definition}(Big Divisor)
    A divisor $L$ is said to be \textit{big} if $nL$ is the sum of an effective divisor and ample divisor for sufficiently large $n$.
\end{definition}

Let's now state Vojta's conjecture in more generality.

\begin{conjecture}(Vojta's Conjecture)
    Let $X/k$ be a non-singular projective variety and $k$ a number field.
    We will also take $A$ to be a big divisor on $X$, 
    $D$ a normal crossings divisor on $X$,
    and $K$ the canonical divisor class on $X$.
    Then for any $\varepsilon > 0$,
    there exists a proper closed subset $Z$,
    depending on all choices made,
    such that for any $P \notin Z$
    \[
        m_S(D, P) + h_{K}(P) \leq \varepsilon h_A(P) + O(1).
    \]
\end{conjecture}

While not much is known with respect to particular cases of Vojta's conjecture,
we have the following result from \cite{McKinnon_2003} which is applicable to our next section.

\begin{theorem}
    Let $C/k$ be a smooth elliptic curve where the group $C(\Q)$ is rank one.
    Suppose that there exists a birational $k$-morphism $f : X \to C \times C$,
    where $X$ is a non-singular, projective $k$-scheme.
    Moreover, suppose that the image of the exceptional set is contained in a finite subset of $(C \times C)(k)$.
    If $L$ is a big divisor on $X$,
    $D$ is the zero divisor with trivial height function,
    and $K_X$ is the canonical divisor,
    then for any $\varepsilon > 0$,
    there exists an effectively computable proper closed subset $Z \subseteq X$ for which $P \in (X \sm Z)(k)$ implies that
    \[
        h_K(P) \leq \varepsilon h_L(P) + O(1).
    \]
\end{theorem}
