\documentclass[12pt]{article}
\usepackage{verbatim}
\usepackage[margin=1 in]{geometry}
\usepackage{rotating} %for \includegraphix[angle=270]
\usepackage{amssymb}
\usepackage{amsthm}
\usepackage{amsmath}
\usepackage{bbm}
%\usepackage{amsmath,amssymb,epsfig,amscd,xy,amsthm}
%\xyoption{all}
%\usepackage[active]{srcltx}
%\CompileMatrices
\usepackage{enumerate}
\usepackage{extarrows}
\usepackage{graphicx}
\usepackage{array}
\usepackage{float}
\usepackage{tikz-cd}

\usepackage{setspace} \doublespacing

\usepackage{user-commands}
\usepackage{theorem-commands}

\begin{document}
    An important abstraction for understanding a given variety are the divisors.
    \begin{definition}[Weil Divisor]
        A \textit{Weil Divisor} on a scheme (integral, regular on dim 1, separated, noetherian) $X$ is a formal sum of codimension 1 closed integral subschemes.
    \end{definition}
    A classic example of a Weil divisor would be a hyperplane in $\bbP^n$,
    often taken as $\{x_0 = 0\}$.

    % Stuff on finiteness
\end{document}
