\documentclass[12pt]{article}
\usepackage{verbatim}
\usepackage[margin=1 in]{geometry}
\usepackage{rotating} %for \includegraphix[angle=270]
\usepackage{amssymb}
\usepackage{amsthm}
\usepackage{amsmath}
\usepackage{bbm}
%\usepackage{amsmath,amssymb,epsfig,amscd,xy,amsthm}
%\xyoption{all}
%\usepackage[active]{srcltx}
%\CompileMatrices
\usepackage{enumerate}
\usepackage{extarrows}
\usepackage{graphicx}
\usepackage{array}
\usepackage{float}
\usepackage{tikz-cd}

\usepackage{setspace} \doublespacing

\usepackage{user-commands}
\usepackage{theorem-commands}

\begin{document}
Next, let's create a topological space for which a given ring will act like a space of functions.
A natural place where functions and rings coincide are polynomial rings,
and indeed this is a motivating example and basis for all of algebraic geometry with the so-called affine spaces.

Consider the polynomial ring $A = k[x_1, \cdots, x_n]$ for some algebraically closed field $k$ and integer $n \geq 1$.
Common sets of interest include the zero sets of some polynomials $f_1, \ldots, f_m \in A$.
However, to speak of a zero set, it is not always clear what exactly this entails for general rings where the notion of ``plugging-in" is not well-defined.
For exactly these cases, it is convenient to note that a point ${(a_1, \ldots, a_n) \in \A_k^n}$,
where $\A_k^n$ can be regarded as $k^n$ for now,
is a common zero of our functions precisely when we have
\[
    f_1 \equiv \cdots \equiv f_m \equiv 0  \mod \langle x_1 - a_1, \ldots, x_n - a_n \rangle.
\]

In fact, even for the non-vanishing points of $\A_k^n$, 
it is easy to see that the value of a given polynomial $f(a_1, \ldots, a_n)$ can be seen to be the unique representative in $k$ modulo the ideal generated by the polynomials $a_1 - x_1, \ldots, a_n - x_n \in A$. 
This can even be generalized to take an integer $n \in \Z$ and use it as a function on the primes of $\Z$ given by
\[
    p \mapsto n \mod p,
\]
and $n$ vanishes precisely at those $p$ which divide it,
or equivalently using ideals, when ${(n) \subseteq (p)}$. 

We are now ready to define a topological space derived from a ring for which the ring itself will provide us a sheaf of functions.

\begin{definition}[Zariski Topology]
    Given a commutative ring $A$, we define the \textit{spectrum of $A$}, written $\Spec(A)$, to be the set of all prime ideals of $A$.
    The \textit{Zariski Topology} will be generated by open subsets given for $f \in A$ of the form
    \[
        D_f = \{ P \in \Spec(A) : f \notin P \},
    \]
    or equivalently, with closed sets given by the ideals $I \subseteq A$ as
    \[
        V(I) = \{ P \in \Spec(A) : I \subseteq P \}.
    \]
\end{definition}

Let's first explore our topology.
Note that the sets $D_f$ do in-fact form a basis for a topology since we may easily satisfy the standard definition.
\begin{enumerate}
    \item Given $P \in \Spec(A)$, 
    we may find some $f \in A \sm P$ so that $P \in D_f$, 
    or else $A = 0$ and $\Spec(A) = \emptyset$.
    \item For $f, g \in A$, we have $D_f \cap D_g = D_{f g}$ as for any $P \in \Spec(A)$, 
    $f g \in P$ if and only if $f \in P$ or $g \in P$.
\end{enumerate}

Examining our other topology, we see that the set $V(I)$ do in-fact form a system of closed sets since we may likewise check a few basic facts.
\begin{enumerate}
    \item It is clear that $V((0)) = \Spec(A)$ and $V(A) = \emptyset$.
    \item Given $I, J \subseteq A$, we find that $V(IJ) = V(I) \cup V(J)$, following from basic facts of prime ideals.
    Note that this implies $V(P)$ is an \textit{irreducible closed set} in the topology as we cannot write $P = IJ$ non-trivially.
    \item For a system of ideals $I_j$ indexed by $j \in S$, $V( \sum_{j \in S} I_j ) = \cap_{j \in S} V(I_j).$
    This one is only slightly less obvious as clearly $\sum_{j \in S} I_j \subseteq P$ implies $V(I_j)$ contains $P$ for all $j \in S$,
    and conversely if $P \in V(I_j)$ for all $j \in S$, then as the smallest ideal containing each $I_j$, $\sum_{j \in S} I_j \subseteq P$.
\end{enumerate}

We may also explain briefly why the two topologies given certainly agree.
Clearly, we have $D_f = \Spec(A) \sm V((f))$ from reading definitions.
Checking that the basis generates this topology as well, 
we may consider an arbitrary open subset $U = \Spec(A) \sm V(I)$ from a given ideal $I \subseteq A$ and a point $P \in U$. 
We may then take any $f \in I \sm P$ (non-empty or else $P \in V(I)$), such that $P \in D_f \subseteq U$.

For an algebraically closed field $k$, let's redefine $\A_k^n$ as $\Spec(A)$, where $A$ is the polynomial ring $k[x_1, \ldots, x_n].$
In this case, we will refer to the closed sets as \textit{algebraic sets} or \textit{affine varieties}.
Recall that $\Spec(A)$ is defined as the set of all prime ideals while $\A_k^n$ was originally defined by the tuples $k^n$.
To make sense of this distinction, note that the primes of $\Spec(A)$ we make an exception for the \textit{closed} points.

A closed point is some $P \in \Spec(A)$ such that $\ol{\{P\}} = \{P\}$.
It is clear that the closed points of $\A_k^n$ will then be maximal ideals of $A$, and hence the following correspondence
\[
    (a_1, \ldots, a_n) \in k^n \biject \langle x_1 - a_1, \ldots, x_n - a_n \rangle \in \A_k^n.
\]
For our other prime ideals which are not maximal, these will then correspond to irreducible algebraic sets as mentioned previously, 
such as curves or surfaces contained in $\A_k^n$ defined by algebraic equations.
Under this correspondence, it is clear that our closed sets $V(I) \subseteq \A^n_k$ are analogous to the points on which all polynomials in $I$ vanish,
also including additional primes for all reducible algebraic sets.

Next, let's investigate how we can generate a sheaf of rings on our topological space. 
As before, we may draw upon our previous examples of $A = k[x_1, \ldots, x_n]$ and $\Z$.
With the former, a natural intuition would be that if you are on an open set $D_f$ for some $f \in A$,
then we could consider rational functions which do not vanish along $D_f$.
That is, if we consider the set of rational functions which vanish only on some subset of $V(f)$ (shorthand for $V((f))$),
then our ring of rational functions becomes fractions of $A$ with denominators not vanishing on $D_f$.

For this, we turn towards Hilbert's Nullstellensatz. 

\begin{theorem}[Hilbert's Nullstellensatz]
    Consider the polynomial ring $A = k[x_1, \ldots, x_n]$ over an algebraically closed field $k$.
    Let $I \subseteq A$ be an ideal and consider $V(I) \subseteq \A_k^n$.
    If $f \in A$ vanishes along $V(I)$, then $f^k \in I$ for some integer $k \geq 1$. 
\end{theorem}

In this context, if we find $g \in A$ vanishes only on a subset of $V(f)$,
then $f$ vanishes along $V(g)$ so $f^k \in (g)$ for some $k \geq 1$. 
Therefore, we may write $f^k = gh$ for some $h \in A$ and
\[
    \frac{1}{g} = \frac{h}{gh} = \frac{h}{f^k}.
\]

Hence, our ring of functions is exactly the functions whose denominators are powers of $f$.
This may be expressed as the localization $A_f$ of $A$ by the multiplicative system $\{ f^n \}_{n \geq 0}$.

Let's now examine what our expected stalks should be by considering the ring $\Z$.
Taking $n \in \Z$ arbitrarily, we may consider $n^{-1}$ defined as a function on the open set $D_n$ since for any $(p) \in D_n$,
$n$ has a multiplicative inverse in the ring $\Z/p\Z$ by the fact $\gcd(n, p) = 1$.
Thus, if we fix a prime $p$ and examine all open basis sets $D_n$ containing $p$,
we find that the pair $\langle D_n, n^{-1} \rangle$ belongs to our stalk.
After considering all such possibilities in this regard, we should expect our stalk to be
\[
    \Z_{(p)} = \left\{
        \frac{a}{b} : a \in \Z, b \in \Z \setminus (p)
    \right\},
\]
which is the localization of $\Z$ by the multiplicative system of integers not contained in $(p)$.


With this, for an arbitrary ring $A$, we determine our sheaf of functions $\cO_{\Spec(A)}$.

\begin{theorem}
    Given a ring $A$ with $X = \Spec(A)$,
    there exists a unique sheaf of rings $\cO_X$ referred to as the \textit{structure sheaf} such that the following three properties hold.
    \begin{enumerate}
        \item Our global sections are the entire ring, given by $\cO_X(X) \cong A$.
        \item For any $f \in A$, $\cO_X(D_f) \cong A_f$.
        \item For any $P \in X$, $\cO_{X, P} \cong A_P$.
    \end{enumerate}
\end{theorem}


\end{document}

% Missing definitions
% - ring localiztion
% - algebraically closed
% - Noetherian ring

