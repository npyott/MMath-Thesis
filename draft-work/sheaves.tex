\documentclass[12pt]{article}
\usepackage{verbatim}
\usepackage[margin=1 in]{geometry}
\usepackage{rotating} %for \includegraphix[angle=270]
\usepackage{amssymb}
\usepackage{amsthm}
\usepackage{amsmath}
\usepackage{bbm}
%\usepackage{amsmath,amssymb,epsfig,amscd,xy,amsthm}
%\xyoption{all}
%\usepackage[active]{srcltx}
%\CompileMatrices
\usepackage{enumerate}
\usepackage{extarrows}
\usepackage{graphicx}
\usepackage{array}
\usepackage{float}
\usepackage{tikz-cd}

\usepackage{setspace} \doublespacing

\usepackage{user-commands}
\usepackage{theorem-commands}

\begin{document}

We begin this section with the basic building block we will use for endowing the various geometrical spaces we will encounter with various rings of locally defined functions.

\begin{definition}[Presheaf]
    A \textit{presheaf} $\cF$ on a topological space $X$ is mapping on the open sets $U \mapsto \cF(U)$, 
    where the sets $\cF(U)$ may either be groups, rings, modules, or other categories.
    Importantly, if given a containment of open sets $V \subseteq U$, 
    there exists a restriction morphism $\rho_{V \to U}: \cF(V) \to \cF(U)$.
    As well, these groups (rings, modules, etc.) and restriction morphisms must satisfy:
    \begin{enumerate}
        \item $\cF(\emptyset)$ is trivial,
        \item $\rho_{U \to U}$ is the identity on $\cF(U)$, and
        \item  Given the containment of open sets $W \subseteq V \subseteq U$, the restriction morphism from $W$ to $U$ is the composition of the other two restriction morphisms, captured by the diagram below.
        \[
            \begin{tikzcd}
                \cF(W) \arrow[d, "\rho_{W \to V}"] \arrow[dr, "\rho_{W \to U}", bend left] \\
                \cF(V) \arrow[r, "\rho_{V \to U}"] & \cF(U)
            \end{tikzcd}
        \]
    \end{enumerate}
\end{definition}

Some other notation we will encounter when dealing with presheaves is that the elements of each local group may be referred to as $\textit{sections}$
and we may alternatively use the notation $\Gamma(U, \cF)$ opposed to $\cF(U)$.
Also, with the idea of functions in mind, given a section $s \in \Gamma(U, \cF)$ and a containment of open sets $V \subseteq U \subseteq X$,
we will almost always prefer to write $s|_V$ opposed to $\rho_{U \to V}(s)$ for the restriction of $s$.

Alongside this, we also want to consider maps between presheaves which, in some sense, respect the underlying algebra.

\begin{definition}[Presheaf map]
    Given two sheaves $\cF$ and $\cG$ on a space $X$, 
    we may define a \textit{map of presheaves} $\varphi : \cF \to \cG$ as a homomorphism $\varphi(U) : \cF(U) \to \cG(U)$ on each open set $U$ which commutes with the restriction maps.
    This relationship is demonstrated with the diagram below for open sets $V \subseteq U \subseteq X$.
    \[
        \begin{tikzcd}
            \cF(U) \arrow[r, "\varphi(U)"] \arrow[d, "\rho_{\cF, U \to V}"] & \cG(U) \arrow[d, "\rho_{\cG, U \to V}"] \\
            \cF(V) \arrow[r, "\varphi(V)"] & \cG(V)
        \end{tikzcd}
    \]
\end{definition}

Composition of sheaf maps is given in the straightforward manner by composition on each underlying homomorphism.
Such a mapping is considered an isomorphism when each homomorphism is itself an isomorphism.
Also, when it is clear which open set a given section $s \in \cF(U)$ belongs to, 
we may elect to write $\varphi(s)$ instead of the much more verbose $\varphi(U)(s)$.

While a presheaf satisfies the job of attaching local data to the various open subsets of our space,
it doesn't have strong enough requirements to glue sections together or guarantee uniqueness.

\begin{definition}[Sheaf]
    A \textit{sheaf} $\cF$ on a topological space $X$ is a presheaf which has certain local properties which mimic the usual notions of functions defined in an open neighbourhood. Specifically, these extra conditions are given as follows.
    \begin{enumerate}
        \item Given an open cover $\cup_{i \in I} U_i$ of some open subset $U \subseteq X$, 
        if a section $s \in \cF(U)$ satisfies $s|_{U_i} = 0 \in \cF(U_i)$, then $s = 0$. 
        That is, we may conclude that the local properties of each section uniquely determine the global properties.
        \item Given an open cover $\cup_{i \in I} U_i$ of some open subset $U \subseteq X$, 
        and some sections $s_i \in \cF(U_i)$ for which $s_i|_{U_i \cap U_j} = s_j|_{U_i \cap U_j}$ for any $i, j \in I$, 
        then there exists some section $s \in \cF(U)$ such that $s_i = s|_{U_i}$.
        That is, given some sections defined on an open cover which agree on overlaps, we may always extend this to construct a section on the broader open subset.
    \end{enumerate}
\end{definition}

One more key concept to keep in mind with a sheaf is the idea of the stalk.
This is a set of local data to a particular point which is obtained by considering the sections on all open neighbourhoods of our point.

\begin{definition}[Stalk]
    Let $P \in X$ be some point and $\cF$ a presheaf. 
    We define the \textit{stalk} $\cF_P$ at the point $P$ by equivalence classes $\langle U, s \rangle$,
    with $U \subseteq X$ open containing $P$ and $s \in \cF(U)$.
    Two pairs $\langle U, s \rangle$ and $\langle V, t \rangle$ are said to be equivalent when there exists some open subset $W \subseteq U \cap V$ such that $s|_W = t|_W$.
\end{definition}

Stalks are very useful tools since for a map of sheaves $\varphi : \cF \to \cG$ on a space $X$,
we get an associated map of stalks $\varphi_P : \cF_P \to \cG_P$.
With the associated maps, we find that $\varphi$ is an isomorphism if and only if $\varphi_P$ is an isomorphism for each $P \in X$.
We will also say a sheaf morphism is injective (surjective) precisely when the associated map on the stalks is injective (surjective) for all $P \in X$.
Note that our injectivity condition is equivalent to each map of sections being injective,
but our surjectivity condition is not equivalent to each map of sections being surjective.

The power of sheaves is demonstrated with the following useful lemma.
Note that sheaf restriction to an open set is done so in the obvious way.

\begin{lemma}[Gluing sheaves]
    Consider an open cover $X = \cup_{i \in I} U_i$, with sheaves $\cF_i$ on $U_i$ for each $i \in I$.
    Suppose for each $i, j \in I$ there exists a sheaf isomorphism $\varphi_{i j} : \cF_i|_{U_i \cap U_j} \to \cF_j|_{U_i \cap U_j}$ which behaves nicely in the following ways for any indices $i, j, k \in I$: 
    \begin{enumerate}
        \item $\varphi_{i i}$ is the identity sheaf map.
        \item $\varphi_{i k} = \varphi_{j k} \circ \varphi_{i j}$ on the intersection $U_i \cap U_j \cap U_k$.
    \end{enumerate}
    Then there exists a sheaf $\cF$ on $X$ such that $\cF_i = \cF|_{U_i}$.
\end{lemma}

\begin{proof}
    % Cite supervisor?
    We may in-fact show the sheaf in question directly.
    Given an open set $U \subseteq X$, we may define the sections as
    \[
        \cF(U) = \{ 
            (s_i)_{i \in I} \in \prod_{i \in I} \cF_i(U_i \cap U): 
            \varphi_{i, j}(s_i|_{U_i \cap U_j \cap U}) = s_j|_{U_i \cap U_j \cap U} \forall i, j \in i
        \},
    \]
    where the operations are done component wise.
    Clearly, by defining the restriction maps componentwise,
    we have a presheaf, so it remains is to validate the sheaf axioms.

    Starting with our uniqueness axiom, let $V \subseteq X$ be arbitrary with open cover $V = \cup_{j \in J} V_j$.
    Suppose for some $s \in \cF(V)$ that $s|_{V_j} = 0$ for each $j \in J$.
    To show that $s = (s_i)_{i \in I}= 0$, we notice that it suffices to show that each component of $s$ is zero.
    For this, we recall that for each $i \in I$, that $U_i$ and $\cF_i$ form a sheaf and $U_i \cap V = \cup_{j \in J} U_i \cap V_j$ is an open cover.
    Thus, since $s|_{V_j} = 0$ for each $j \in J$, we find $s_i|_{U_i \cap V_j} = 0$ and so $s_i = 0$ by our sheaf axiom.
    Therefore, it must be that $s = 0$ as well.

    Next, similarly take $V \subseteq X$ open with the same open cover as before. 
    Suppose now that we are given some sections $s^{(j)} = (s^{(j)}_i)_{i \in I} \in \cF(V_j)$ for which $s^{(j)}|_{V_j \cap V_{j'}} = s^{(j')}|_{V_j \cap V_{j'}}$ for all $j, j' \in J$.
    As before, we wish to use our sheaf axioms to determine for each $i \in I$ some $s_i \in \cF_i(U_i \cap V)$ so that for all $j \in J$,
    \[
        (s_i)_{i \in I}|_{V_j} = s^{(j)}
        \iff s_i|_{U_i \cap V_j} = s^{(j)}_i.
    \]
    As before, fixing $i \in I$, note that $\bigcup_{j \in J} U_i \cap V_j$ is an open cover of $U_i \cap V$.
    Thus, since for each $j, j' \in J$ we have $s^{(j)}|_{V_j \cap V_{j'}} = s^{(j')}|_{V_j \cap V_{j'}}$, we must then have
    \[
        s^{(j)}_i|_{U_i \cap V_j \cap V_{{j'}}}
        = s_i^{(j')}|_{U_i \cap V_j \cap V_{{j'}}}.
    \]
    Therefore, it follows that there exists some section $s_i \in \cF(U_i \cap V)$ such that $s_i|_{U_i \cap V_j} = s^{(j)}_i$.
    Lastly, we also remark that for each $i, k \in I$ that $\varphi_{ik}(s_i|_{U_i \cap U_k \cap V}) = s_k|_{U_i \cap U_k \cap V})$.
    This is since $\cup_{j \in J} V_j \cap U_i \cap U_k$ is an open cover and we have on each restriction $V_j \cap U_i \cap U_k$ that
    \begin{align*}
        \varphi_{ik}(s_i|_{U_i \cap U_k \cap V})|_{V_j \cap U_i \cap U_k}
        & = \varphi_{ik}(s^{(j)}_i|_{U_i \cap U_k \cap V_j}) \\
        & = s^{(j)}_k|_{U_i \cap U_k \cap V_j} \\
        & =(s_k|_{U_i \cap U_k \cap V})|_{V_j \cap U_i \cap U_k}.
    \end{align*}

    Finally, we also briefly mention why $\cF_i \cong \cF|_{U_i}$ for any particular $i \in I$.
    The isomorphism from $\cF|_{U_i}$ to $\cF_i$ is given by projection onto the corresponding coordinate.
    The mapping is clearly injective by the fact that the other components are given by isomorphisms on intersections. 
    Also, the mapping is surjective since given $U \subseteq U_i$, if $s_i \in \cF_i(U)$, then we can consider
    \[
        (\varphi_{ik}(s_i|_{U \cap U_k}))_{k \in I} \in \cF|_{U_i}(U),
    \]
    which clearly projects onto our given section.
    \qedhere
    
\end{proof}

While this clearly demonstrates the utility of the sheaf axioms, 
we are often left with only a presheaf.
However, we may in-fact uniquely extend any given presheaf to a sheaf by examining the stalks.

\begin{theorem}[Sheaf associated to a presheaf]
    Let $\cF$ be a presheaf on a topological space $X$.
    There exists a sheaf $\cF^+$ and map $\theta : \cF \to \cF^+$, 
    unique up to isomorphism, such that $\cF_P \cong \cF^+_P$ for any $P \in X$.
    Moreover, we have the following universal property. 
    
    Given a sheaf $\cG$ and a map $\varphi : \cF \to \cG$, 
    there exists a unique morphism of sheaves $\psi : \cF^+ \to \cG$ such that the following diagram commutes.
    \[
        \begin{tikzcd}
            \cF \arrow[r, "\varphi"] \arrow[d, "\theta"] & \cG \\
            \cF^+ \arrow[ur, "\psi"]
        \end{tikzcd}
    \]
\end{theorem}

One way to view our associated sheaf is to define it as the gluing of sections from $\cF$.
In fact, we may verify that if we set 
\[
    \cF^+(U) = \{\{(U_i, s_i)\}_{i \in I} : 
        U = \cup_{i \in I} U_i, 
        s_i \in \cF(U_i), 
        s_i|_{U_i \cap U_j} = s_j|_{U_i \cap U_j}
    \},
\]
with $\theta : \cF \to \cF^+$ given by $\theta(s) = \{(U, s)\}$ for any open $U \subseteq X$.
There is also an implicit equivalence relationship when two sets of pairs agree on all mutual overlaps.
Note that restriction is done pairwise and we may simply discard any empty restrictions,
while our algebra is done on intersection pairs between the two open covers.
We also define our stalks to be given by taking any of the open set and section pairs.
As they agree on overlaps, this is well-defined. 


If we have a morphism $\varphi : \cF \to \cG$ with $\cG$ a sheaf on $X$,
then clearly we must have $\psi : \cF^+ \to \cG$ given by
\[
    \psi(\{(U_i, s_i)\}_{i \in I})|_{U_i} = \psi(\{(U_i, s_i)\}) = (\psi \circ \theta)(s_i) = \varphi(s_i).
\]
This definition makes sense as $\cG$ is a sheaf,
and so there must exist some $t \in \cG(U)$ such that $t|_{U_i} = \varphi(s_i)$ as each $\varphi(s_i)$ necessarily agree on overlap.

Note that with the sheaf morphism $\psi : \cF^+ \to \cG$, we have satisfied the universal property and is hence isomorphic to the claimed sheaf.
We also note that it is clear that the stalks are isomorphic.

For one last important aspect of sheaves which we will make use of,
consider two topological spaces $X$ and $Y$ with sheaves $\cF$ and $\cG$ respectively.
Given a continuous map $f : X \to Y$, 
we are able to consider how $\cF$ may act as a sheaf on $Y$ and likewise for $\cG$ on $X$.

We consider the \textit{pushforward} $f_*\cF$ on $Y$ as for each $U \subseteq Y$, $(f_*\cF)(U) = \cF(f^{-1}(U))$.
Likewise, we are able to consider the \textit{inverse image shead} $f^{-1}\cG$ on $X$ by equivalence classes of ordered pairs given below
\[
    (f^{-1}\cG)(U) = \{ (V, s) : f(U) \subseteq V, s \in \cG(V) \}
     \big / \sim
\]
with $(V, s) \sim (W, t)$ whenever there exists some open neighbourhood $O \subseteq V \cap W$ also containing $f(U)$ for which $s|_O = t|_O$.
\end{document}

% Missing definitions
% - ring localiztion
% - algebraically closed
% - Noetherian ring

