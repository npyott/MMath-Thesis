\documentclass[12pt]{article}
\usepackage{verbatim}
\usepackage[margin=1 in]{geometry}
\usepackage{rotating} %for \includegraphix[angle=270]
\usepackage{amssymb}
\usepackage{amsthm}
\usepackage{amsmath}
\usepackage{bbm}
%\usepackage{amsmath,amssymb,epsfig,amscd,xy,amsthm}
%\xyoption{all}
%\usepackage[active]{srcltx}
%\CompileMatrices
\usepackage{enumerate}
\usepackage{extarrows}
\usepackage{graphicx}
\usepackage{array}
\usepackage{float}
\usepackage{tikz-cd}

\usepackage{setspace} \doublespacing

\usepackage{user-commands}
\usepackage{theorem-commands}

\begin{document}
    As with our initial result from BCZ[replace], 
    a key element to characterizing our result was the arithmetic complexity of a given integer.
    While it is simple in the case of integers, 
    we note that we may easily and naturally extend our notion of arithmetic complexity.

    As an example, consider Roth's theorem mentioned previously.

    \begin{theorem}[Roth's Theorem]
        Given an algebraic number $\alpha \in \R$, for any $\varepsilon > 0$,
        there exists at most finitely many co-prime integers $p, q \in \Z$ such that
        \[
            \left|\alpha - \frac{p}{q}\right| < \frac{1}{|q|^{2 + \varepsilon}}.
        \]
    \end{theorem}

    In this example, we have a notion on the arithmetic complexity of a fraction $p/q$ by taking its denominator.
    It should be noted that only the denominator is considered in this expression of Roth's theorem due to the common analysis of reals in the interval $[0, 1)$.
    For completeness, we may more symmetrically examine $\max(|p|, |q|)$.

    However, looking at only this norm on the rationals is only half of the picture.
    We recall that for each prime $p \in \Z$, 
    that $\Z_{(p)}$ is a discrete valuation ring with $p\Z_{(p)}$ the unique maximal ideal.
    From this, we obtain a function
    \[
        \ord_p : \Z_{(p)} \setminus \{0\} \to \N
    \]
    such that $\ord_p(ab) = \ord_p(a) + \ord_p(b)$, $\ord_p(p) = 1$, and $\ord_p(x) = 0$ for any $x \notin p\Z_{(p)}$.
    Clearly, this corresponds to our notion of the unique power of $p$ in the unique prime factorization of a given integer.
    
    Extending this to the field of fractions by preserving our additive rule,
    we obtain a new \textit{$p$-adic norm} on our rational numbers given for $a \in \Q \setminus \{0\}$ as $|a|_p = p^{-\ord_p(a)}$ and $|0|_p = 0$.
    We may also note one interesting quirk when verifying our triangle inequality that we rather find an \textit{ultrametric inequality}.
    For any $a, b \in \Q$, we may show that $|a + b|_p \leq \max\{|a|_p, |b|_p\}$,
    which easily follows as a result of unique prime factorization.
    Due to this, the natural numbers are obviously bounded,
    and hence we refer to such absolute values as \textit{non-archimedean} while our usual absolute value is \textit{archimedean}.
    We refer to the collection of norms on $\Q$ by the set $M_\Q$ and use the notation $|\cdot|_v$ to represent the $v$-norm.

    With this, we are ready to define the height of a rational number $r \in \Q$. 
    Looking instead at the rational point $(r : 1) \in \bbP^1$, 
    which can be re-expressed with co-prime integer coordinates $(a : b)$,
    we set 
    \[
        H_\Q(a : b) = \prod_{v \in M_\Q} \max\{|a|_v, |b|_v\}.
    \]
    While this perfectly coincides with our original definition for $\gcd(a, b) = 1$,
    we also note that this norm is perfectly invariant to the scaling of coordinates as we'd expect for $\bbP^1$.
    This definition can be easily extended to higher dimensional projective spaces by taking a maximum over more coordinates.

    Next, let's mention how we can extend our notion of algebraic complexity to finite field extensions $k/\Q$.
    We begin with our non-archimedean norms, denoted $M_k^0$.
    Setting $R_k$ as our ring of integers and fixing a prime ideal $P \in \Spec(R)$, 
    note that $R_{k,P}$ is likewise a local ring with a valuation.
    Therefore, we may similarly obtain a homomorphism $\ord_P : k^* \to \Z$.

    In order to now find an exponential base,
    we will use $N_{k/\Q}(P) = |R_k / P|$.
    With this, we take for $x \in k^*$,
    \[
        \|x\|_P = N_{k / \Q}(P)^{-\ord_P(x)},
    \]
    and of course $\|0\|_P = 0$.
    In this way, as $R_k / P$ is a finite field and $\Spec(R_k) \to \Spec(\Z)$ puts $R_k$ over some prime $p \in \Z$,
    we may define all of our p-adic norms on $k$. 
    Pay special attention to this case as if $p \in \Z$ is still prime over $R_k$,
    then $N_{k / \Q}(p) = p^{[k : \Q]}$.

    However, we also usually have more than one archimedean norm on $k$ as well.
    In fact, we obtain one for each real embedding of $k \to \R$,
    as well as one for each pair of conjugate complex embedding $k \to \C$.
    Once a particular embedding $\rho : k \to \C$ is chosen,
    we simply take for $x \in k$
    \[
        \|x\|_\rho = |\rho(x)|^{n_\rho},
    \]
    where $n_{\rho} = [\F : \R]$ with $\F = \C$ for a complex embedding and $\F = \R$ for a real embedding.
    We commonly denote the archimedean norms on $k$ as $M^\infty_k$.

    Finally, with an understanding of the set of places $M_k$ of a finite field extension $k/\Q$,
    we may define our relative heights with respect to a choice of coordinates on $\bbP^n$.
    Note that $k$-rational points in $\bbP^n$ may be thought of as any point who has some choice of coordinates all lying $k$. 

    \begin{definition}
        Let $k/\Q$ be a finite field extension.
        For a point $P = (a_0, \ldots, a_n) \in \bbP^n(k)$,
        we define the \textit{multiplicative height} as
        \[
            H_k(P) = \prod_{v \in M_k} \max\{\|a_0\|_v, \ldots, \|a_n\|_v\}.
        \]
        We also define the \textit{logarithmic hieght} as $h_k(P) = H_k(P)$.
    \end{definition}

    We also attribute a more general height on our projective space.

    \begin{definition}
        The \textit{absolute multiplicative height} $H: \bbP^n(\ol\Q) \to [1, \infty)$ is given by 
        \[
            H(P) = H_k(P)^{\frac{1}{[k : \Q]}},
        \]
        where $P \in \bbP(k)$ and choice of finite extension $k/\Q$ is arbitrary.
        We similarly define the \textit{absolute logarithmic height} $h : \bbP^n(\ol\Q) \to [0, \infty)$ as $h(P) = \log H(P)$.
    \end{definition}

    Let's show that these height functions are well-defined irrespective of homogeneous coordinates or choice of field for our absolute heights.
    First, we recount two important facts from Neukirch as well as Silverman and Hindry.

    \begin{lemma}[Degree formula]
        Let $k_2 / k_1$ be a finite extension of number fields.
        We write $w | v$ for $w \in M_{k_2}$ to denote that $|\cdot|_w$ restricted to $k_1$ is the same as $|\cdot|_v$,
        as well as denoting $k_{2, w}$ and $k_{1, v}$ as the completion of these fields with respect to the metrics induced by the respective absolute values.
        Then for any absolute value $v \in M_{k_1}$ on the base field,
        \[
            \sum_{\substack{
                w \in M_{k_2} \\
                w | v
            }} [k_{2, w} : k_{1, v}] = [k_2 : k_1].
        \]
    \end{lemma}

    \begin{lemma}[Product formula]
        Let $k/\Q$ be a finite field extension.
        For any $x \in k^*$, we have
        \[
            \prod_{v \in M_k} \|x\|_v = 1.
        \]
    \end{lemma}

    \begin{proof}
        We note that this is clearly true for $k = \Q$ as each $p$-adic norm will divide out by the part of a rational lying over $p$.
        To verify for a field extension $k/\Q$ with ring of integers $R_k$,
        consider by the fact that $R_k$ is a Dedekind domain that for any $x \in k^*$,
        \[
            xR_k = \prod_{i = 1}^r P_i^{\ord_{P_i}(x)}.
        \]
        By taking norms, we find
        \[
            |N_{k/\Q}(x)| = N_{k / \Q}(xR_k) = \prod_{i = 1}^r N_{k / \Q}(P_i)^{\ord_{P_i}(x)}.
        \]
        Therefore, once we consider a particular prime $p \in \Z$,
        as each $p$-adic norm corresponds to the primes lying over $p$,
        we must have
        \begin{align*}
            |N_{k/\Q}(x)|_p 
            & = \prod_{i = 1}^r |N_{k / \Q}(P_i)^{\ord_{P_i}(x)}|_p \\
            & = \prod_{\substack{P \in \Spec(R_k) \\ x, p \in P}} N_{k / \Q}(P)^{-\ord_P(x)} \\
            & = \prod_{\substack{v \in M_k \\ v | p}} \|x\|_v.
        \end{align*}
        This similarly holds for archimedean extensions of the usual absolute value on $\Q$ as well.
        In the following calculation, 
        note that the squaring of absolute values associated to complex embeddings can be accounted for as a product of the embedding and its conjugate,
        and that the norm of an algebraic number of $\Q$ is often defined as the product of all embeddings.
        Putting this together grants us
        \[
          \prod_{v | \infty} \|x\|_v 
          = \prod_{\sigma : k \to \C} |\sigma(x)|_\infty
          = |N_{k/\Q}(x)|_\infty.  
        \]
        With our identity true for all places $M_\Q$, we conclude
        \begin{align*}
            \prod_{v \in M_k} \|x\|_v
            & =  \prod_{v_0 \in \Q} \prod_{\substack{v \in M_k \\ v | v_0}} \|x\|_v \\
            & = \prod_{v_0 \in \Q} |N_{k/\Q}(x)|_{v_0} \\
            & = 1.
        \end{align*}
    \end{proof}

    Now, we may confirm that our height functions are well-defined.

    \begin{proposition}
        Fix a finite field extended $k/\Q$ and fix some point $P \in \bbP^n(k)$.
        Then the height $H_k(P)$ is independent of choice of homogeneous coordinates.
        Moreover, if $k'/ k$ is an additional finite extension,
        then $H_{k'}(P) = H_k(P)^{[k' : k]}$ and hence our absolute height is well-defined.
    \end{proposition}

    \begin{proof}
        We begin by verifying our first claim. 
        Choose some coordinate $P = (a_0, \ldots, a_n)$ and consider some non-zero scalar $c \in k^*$.
        Then, using our product formula,
        \begin{align*}
            \prod_{v \in M_k} \max \{\|ca_0\|_v, \ldots, \|ca_n\|_v\}
            & = \prod_{v \in M_k} \|c\|_v \max\{\|a_0\|_v, \ldots, \|a_n\|_n\} \\
            & = \left(
                \prod_{v \in M_k} \|c\|_v
            \right) \left(
                \prod_{v \in M_k}\max \{\|a_0\|_v, \ldots, \|a_n\|_v\}
            \right) \\
            & = \prod_{v \in M_k}\max \{\|a_0\|_v, \ldots, \|a_n\|_v\}.
        \end{align*}
        Next, let's inspect $H_{k'}(P)$ given below as 
        \[
            H_{k'}(P) = \prod_{v \in M_{k'}} \max\{\|a_0\|_v, \ldots, \|a_n\|_v\}.
        \]
        With our degree formula in mind, 
        let's break down this product by absolute values on $k'$ which lie over a particular absolute value on $k$,
        \[
            H_{k'}(P) = \prod_{v \in M_k} \prod_{\substack{ w \in M_{k'} \\ w | v}}
            \max\{\|a_0\|_w, \ldots, \|a_n\|_w\}.
        \]
        To proceed, let's consider the cases of an archimedean or non-archimedean absolute value.
        If $w \in M_{k'}$ extending $v \in M_k$ corresponds to some embedding $\sigma' : k' \to k'_w$,
        then $v$ would correspond to $\sigma : k \to k_v$,
        where $k'_w, k_v = \C$ or $\R$ and $k_v \subseteq k'_w$.
        Hence, for any $x \in k$,
        \[
            \|x\|_w 
            = |\sigma'(x)|^{[k'_w : \R]}
            = |\sigma(x)|^{[k'_w : k_v] \cdot [k_v : \R]}
            = \|x\|_v^{[k'_w : k_v]}.
        \]
        Next, let $P' \in \Spec(R_{k'})$ lie over $P \in \Spec(R_k)$.
        Then, as we may view $R_k / P$ as a subfield of $R_{k'} / P'$,
         we may calculate
        \begin{align*}
            N_{k' / \Q}(P')
            & = |R_{k'} / P'| \\
            & = |R_k / P|^{[R_{k'} / P'] : [R_k : P]} \\
            & = N_{k / \Q}(P)^{[R_{k'} / P'] : [R_k : P]}.
        \end{align*}
    \end{proof}

\end{document}
