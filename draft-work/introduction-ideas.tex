\documentclass[12pt]{article}
\usepackage{verbatim}
\usepackage[margin=1 in]{geometry}
\usepackage{rotating} %for \includegraphix[angle=270]
\usepackage{amssymb}
\usepackage{amsthm}
\usepackage{amsmath}
\usepackage{bbm}
%\usepackage{amsmath,amssymb,epsfig,amscd,xy,amsthm}
%\xyoption{all}
%\usepackage[active]{srcltx}
%\CompileMatrices
\usepackage{enumerate}
\usepackage{extarrows}
\usepackage{graphicx}
\usepackage{array}
\usepackage{float}
\usepackage{tikz-cd}

\usepackage{setspace} \doublespacing

\usepackage{user-commands}
\usepackage{theorem-commands}

\begin{document}

A persisting and fascinating example from which a lot of very difficult yet interesting questions may be asked is by considering integer sequences of the form $(d_n = a^n - b^n)_{n \geq 1}$ for some integers $a, b \in \Z$.
From its very construction, some natural questions arise concerning divisibility given that $d_n | d_m$ whenever $n | m$,
as well as the fact that for $b = 1$, the sequence $(a^n - 1)_{n \geq 1}$ is always one off from increasingly greater perfect power.

% perhaps some other stuff goes here
% Maybe investigate:
% - If (a^n - 1) | (b^n - 1) for infinitely many then b = a^k
% - Results on largest prime factors of a^n - 1

In a paper by BCZ[replace], the following interesting relationship between neighbouring sequences $a^n - 1$ and $b^n - 1$ is shown in regards to their greatest common divisor.

\begin{theorem}
    Given two multiplicatively independent integers $a, b \in \Z$ at least $2$, and some $\varepsilon > 0$,
    we may eventually determine some $N \geq 1$ such that for $n \geq N$ thereafter
    \[
        \gcd(a^n - 1, b^n - 1) < \exp(\varepsilon n).
    \]
\end{theorem}

% Cite Diophantine Approximation prof from presentation
When dealing with such sparse sequences,
most methods of counting factors such as sieves will not suffice and instead we turn towards Diophantine approximation.
In no exception, the proof produced by the authors here is first carried out with Schmidt's Subspace Theorem, 
which is a very powerful result in modern Diophantine approximation.

Our trick for analyzing these points will be to acknowledge that powers of integers fall into only so many primes,
exactly the finite primes which divide our initial integers.
From here, we are able to analyze the height relative to this finite set of primes and force a contradiction by using the tools of Diophantine geometry to conclude that infinitely many integers belong where only finitely many should be.
This technique is standard such as the use of Roth's theorem on the so-called $S$-unit equation,
or using Siegel's Theorem on elliptic curves to conclude finitely many integer points.

With this in mind, it is natural to explore generalizations of the BCZ[replace] result to further geometries as well as related problems.
One such tool which generalizes the Subspace Theorem and Siegel's Theorem is a main conjecture of Vojta.
By using algebraic heights to relate GCDs and Vojta's conjecture, 
we will use blowups of projective space and products of elliptic curves to find various conditional results by Silverman.
To obtain more context on Vojta's conjecture applied to blowups,
we will also use McKinnon's paper to show a particular case of blowups of products of elliptic curves.

This thesis is heavy in background as most of the tools used by McKinnon and Silverman rely on the extensive machinery of algebraic geometry. 
Our first objective will be to cover the main definitions and theorems used to determine the canonical class on blowups and fibred products right from the definition of a sheaf.
The machinery built up along the way will also be used to briefly cover Weil's height machine, 
some basic properties of the group law on elliptic curves,
and integer models of smooth varieties.



\end{document}

% Missing definitions
% - multiplicatively independent
