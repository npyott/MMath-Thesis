With much of the geometry behind us,
let's turn to another interesting curve,
which encodes a similar level of novel arithmetic information on a level comparable to $\bbP^1$.
To understand \textit{elliptic curves},
we begin with a famous result of projective curves.


\begin{theorem}[Riemann-Roch for Curves]
    Let $C$ be a non-singular projective curve with canonical sheaf $K_C$.
    For any $D \in \Div(C)$,
    we denote
    \[
        \ell(D) = \dim_k L(D).
    \]
    Then there exists an integer $g \geq 0$,
    referred to as the \textit{genus},
    such that for any $D \in \Div(C)$,
    \[
        \ell(D) - \ell(K_C)
        = \deg(D) - g + 1
    \]
\end{theorem}

\begin{proof}
    Refer to Theorem IV.1.3 of \cite{Hartshorne_2013}.
\end{proof}

Note that when $D = 0$, we have $\deg(D) = 0$ and $\ell(D) = 1$,
so that $\ell(K_C) = g$ after re-arrangement.

\begin{proposition}
    Let $C/k$ be a smooth projective curve of genus 1, 
    then there exists $a_1, \ldots, a_6 \in k$ such that we may embed $C$ into $\bbP_k^3$ by
    \[
        Y^2Z + a_1 XYZ + a_3 YZ^2
        = X^3 + a_2 X^2 Z + a_4 X Z^2 + a_6 Z^3
    \]
\end{proposition}

\begin{proof}
    To begin,
    notice by Riemann-Roch that the canonical divisor is trivial as it must have dimension 1.
    Next, choose a point $O \in E$, consider $\ell(nO)$ for $n \geq 0$.
    Since $nO$ is effective, there is certainly no $f \in K(C)^*$ for which $\rmDiv(f) - nC$ has no poles anywhere,
    and so $\ell(-nO) = 0$.
    Therefore, now applying Riemann-Roch again, we may conclude $\ell(nO) = n$ for all $n \geq 1$

    Next, let's go through the construction of $L(nO)$ for each $n \geq 0$.

    When $n = 1$, we simply have $L(nO) = \Span_k\{1\}$.
    
    For $n = 2$, we may then obtain some $x \in K(C)^*$ such that $2O + \rmDiv(x) \geq 0$.
    From this point, we also note that $2mO + \rmDiv(x^m) \geq 0$. Now our space is $L(2O) = \Span_k\{1, x\}$.

    For $n = 3$, we may necessarily obtain an additional $y \in K(C)^*$ such that $3O + \rmDiv(y) \geq 0$.
    From this point, we also note that $3mO + \rmDiv(y^m) \geq 0$. Now our space is $L(3O) = \Span_k\{1, x, y\}$.

    At $n = 4$, we finally may obtain a new rational function from our previous basis.
    It then suffices to take $L(4D) = \Span_k \{1, x, y, x^2 \}$.

    Likewise, at $n = 5$, the section $xy \in K(C)^*$ will suffice,
    bringing us up to $L(5D) = \Span_k \{1, x, y, x^2, xy \}$.

    Finally, for $n = 6$, we have reached a conflict since we know of seven sections in $K(C)^*$ which belong to $L(6O)$, but $\ell(6O) = 6$.
    Thus, we may obtain some $a_1, \ldots, a_6 \in k$ such that
    \[
        y^2 + a_1 xy + a_3 y
        = x^3 + a_2 x^2 + a_4 x + a_6.  
    \]

    With these rational functions, we take $\varphi: C \to \bbP^2$ to be the rational map given as
    \[
        \varphi(P) = (x(P) : y(P) : 1).
    \]
    It can be deduced from Theorem A.4.2.4 of \cite{Silverman_Hindry_2013} that $3O$ is very ample, 
    and hence we obtain our desired result.

\end{proof}

Assuming the defining field is not of characteristic 2 or 3,
by following the proof of Theorem A.4.4.1 in \cite{Silverman_Hindry_2013},
We may also write our defining equation in \textit{Weierstrass form} by
\[
    Y^2Z = X^3 + aXZ^2 + bZ^3.
\]
The fact that $C$ is non-singular is equivalent to the condition that $4a^3 + 27b^2 \neq 0$,
an assumption we will often make.
We may also use this form of the equation to analyze a rational point $P = (x_P : y_P : 1) \in C(\Q)$.
By substituting into our Weierstrass equation and clearing denominators,
we may determine that we can write $x_P = A_P / D_P^2$ and $y_P = B_P / E_P^3$ in reduced form.


Fix a point $O \in C$ and consider a mapping on the closed points of $C$ to $\Pic^0(C)$,
divisors of degree zero in the equivalence class,
by $P \mapsto P - O$.
Note that for any pair of points $P, Q \in C$,
by Bézout's Theorem (Corollary I.7.8 of \cite{Hartshorne_2013}), the line through $P$ and $Q$ passes through a third point $R$,
up to multiplicity.
If we take $L$ to be the rational function associated to this line,
then by dividing by the line $T$ which vanishes to order 3 at $O$,
we obtain in our Picard group that
\[
    \rmDiv(L/T) + 3O = P + Q + R.
\]
Therefore, as $(P - O) + (Q - O) + (R - O)$ is the identity in $\Pic^0(P)$,
this grants us a group law that $P + Q = -R$.