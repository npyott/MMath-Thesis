With some sense to the geometry of the fundamental affine spaces,
we are ready to consider the generalized concept of a scheme.

\begin{definition}[Scheme]
    An \textit{affine scheme} is a space $X$ with structure sheaf $\cO_X$ which is isomorphic to $\Spec(A)$ and structure sheaf $\cO_{\Spec(A)}$,
    in the sense that $\Spec(A)$ is homeomorphic to $X$ and $\cO_X$ is isomorphic to the pushforward of $\cO_{\Spec(A)}$ as sheaves on $X$.

    A \textit{scheme} is a topological space $X$ with sheaf of rings $\cO_X$ such that for any point $P \in X$,
    there exists some open neighbourhood $U \subseteq X$ of $P$ such that $(U, \cO_X|_U)$ is an affine scheme.

    A morphism of schemes $(f, f^\#) : (X, \cO_X) \to (Y, \cO_Y)$ is a continuous map $f : X \to Y$ and sheaf map $f^\# : \cO_Y \to f_*\cO_X$.
    Additionally, we require that when restricted to affine open subschemes $(U, \cO_X|_U)$ and $(V, \cO_Y|_V)$ such that $U \subseteq f^{-1}(V)$,
    that the restricted map $(f|_U, f^\#|_U) : (U, \cO_X|_U) \to (V, \cO_Y|_V)$ is given by a ring homomorphism $\Gamma(V, \cO_Y|_V) \to \Gamma(U, \cO_X|_U)$.
\end{definition}

Let's consider our first fundamental example of a scheme which is not necessarily affine.
Consider a graded ring $S = \bigoplus_{d \in \Z} S_d$,
where for $d \in \Z$, $S_d$ denotes an additive group of \textit{homogeneous} elements of the same degree,
and we require that $S_d \cdot S_e \subseteq S_{d + e}$ for any integers $d,e \in \Z$.
We will also denote $S_+$ to be all elements of positive degree in $S$.
Lastly, we will say that an ideal is a \textit{homogeneous ideal} when it is generated by homogeneous elements.

With this, we define $\Proj(S)$ to be the space of all homogeneous prime ideals which do not contain all of $S_+$.
To induce a topology on $\Proj(S)$,
our closed sets will be given for each homogeneous ideal $I \subseteq S$ as
\[
    V_+(I) = \{ P \in \Proj(S) : I \subseteq P \}.
\]
Just as in the affine case, we have same familiar rules of arbitrary intersections and finite unions of these closed sets. 
Similarly, we also have distinguished open subsets given for each homogeneous element $f \in S_+$ as $D_+(f) = \Proj(S) \sm V_+((f))$.

Let's now put a sheaf of rings on $\Proj(S)$ to define a scheme.

\begin{theorem}
    Let $S$ be a graded ring and denote $\Proj(S)$.
    There exists a sheaf of rings $\cO$ on $\Proj(S)$ satisfying the following:
    \begin{enumerate}
        \item $(\Proj(S), \cO)$ defines a scheme.

        \item For any $P \in \Proj(S)$, $\cO_{X, P} \cong S_{(P)}$,
        where $S_{(P)}$ is the subring of elements of degree zero in the localized graded ring $S_P$.

        \item For any $f \in S_+$, we have $(D_+(f), \cO_X|_{D_+(f)})$ is an affine scheme isomorphic to $\Spec(S_{(f)})$,
        where $S_{(f)}$ is the subring of $S_f$ of elements of degree zero.
    \end{enumerate}
\end{theorem}
\begin{proof}
    Refer to Proposition II.2.5 of \cite{Hartshorne_2013}.
\end{proof}

A very important example following from the previous theorem is projective space.
Given some ring $A$, we define \textit{projective $n$-space over $A$} to simply be 
\[
    \bbP^n_A = \Proj(A[x_0, \ldots, x_n]).
\]
Over our projective space, notice that for each $0 \leq i \leq n$,
we may cover the entire space with affine patches
\[
    D_+(x_i) = \Spec(A[\tfrac{x_0}{x_i}, \ldots, \tfrac{x_n}{x_i}]).
\]
Moreover, when considering $\bbP^n_k$ for an algebraically closed field $k$,
the closed points of $\bbP^n_k$ are given by \textit{homogeneous coordinates} $(a_0 : \cdots : a_n)$ corresponding to the maximal ideals
\[
    \langle a_i x_j - a_j x_i \rangle_{0 \leq i, j \leq n} \subseteq k[x_0, \ldots, x_n],
\]
which is described locally on $D_+(x_i)$ for $0 \leq i \leq n$ as $(\tfrac{a_0}{a_i}, \ldots, \tfrac{a_n}{a_i}) \in \A_k^n$.
Notice that in both cases our homogeneous coordinates are invariant to scaling by $\alpha \in k^*$.
With the latter property, $\bbP_k^n$ is sometimes thought of as the space of lines through the origin in $\A_k^{n + 1}$,
and the open affine sets $D_+(x_i) \cong \A_k^n$ is the projection of each line onto the plane $\{ x_i = 1 \}$.
Also under this interpretation, 
the lines contained within $\{ x_i = 0 \}$ that never intersect the plane are thought of as being contained in the hyperplane at infinity.
This inspires an obvious interpretation,
at least for the closed points,
that for any $n \geq 1$,
\[
    \bbP_k^n = \A_k^n \cup \bbP_k^{n - 1} = \A_k^n \cup \A_k^{n - 1} \cup \cdots \cup \A_k^1 \cup \A_k^0.
\]

Moving on from projective space for now,
let's consider the closed subschemes of $\Proj(S)$ for a graded ring $S$.
As before, consider a map $\varphi : S \to T$ which is surjective and preserves degree.
Clearly, the preimage of homogeneous prime ideals in $\Proj(T)$ will likewise be homogeneous prime ideals,
and by surjectivity and degree preservation, any homogeneous prime ideal whose preimage contains all of $S_+$ must therefore contain all of $T_+$.
Thus, we obtain a map $f: \Proj(T) \to \Proj(S)$ which is injective for the same reason as the affine case. 
We also obtain surjective maps on local rings and open affine pieces defined by $\phi$,
and so we see that $\Proj(T) \cong \Proj(S/\ker \varphi)$ is a closed subscheme identified with $V_+(\ker \varphi)$.
These facts follow from Exercise II.2.14  and Exercise II.3.12 of \cite{Hartshorne_2013}.

So far, we've kept our schemes and rings quite general in our setup,
but our examples are often finitely generated $k$-algebras for an algebraically closed field $k$.
While these rings provide very nice geometric intuition in the ways we've just described,
we can take some caution and state which properties of our rings and spaces we would like or require.

Another type of scheme we will look towards will be integral schemes.
A scheme $X$, with structure sheaf $\cO_X$,
is said to be \textit{integral} when $\cO_X(U)$ is an integral domain for all open subsets $U \subseteq X$.
We also note that $\Spec A$ is integral if and only if $A$ is an integral domain.
To understand this further,
let's take a quick look at some consequences of this.

The first deduction we can make is that an integral scheme $X$ is \textit{irreducible},
and not the union of any two proper closed subsets.
Otherwise, we would be able to find the complements of these closed subsets,
which we will denote as $U$ and $V$,
would necessarily have trivial intersection.
Following the diagram below resulting from the open cover $X = U \cup V$,
\[
    \begin{tikzcd}
        & \cO_X(U) \arrow[dr] \\
        \cO_X(X) \arrow[ur] \arrow[dr] & & \cO_X(U \cap V) = 0 \\
        & \cO_X(V) \arrow[ur]
    \end{tikzcd}
\]
we must have that $\cO_X(X) \cong \cO_X(U) \times \cO_X(V)$, which is certainly not an integral domain,
with the restriction maps granting the isomorphism $ s \mapsto (s|_U, s|_V)$. 
By applying our sheaf axiom regarding the vanishing of a global section with respect to vanishing of local sections this open cover,
it is clear why the proposed map is injective.
Likewise, using the sheaf axiom regarding the existence of a global section from local sections, this open cover with trivial overlap also explains why the map is surjective.


Note that irreducibility of a closed subset $Z \subseteq X$ implies that it has a \textit{generic point}.
By taking $U \cong \Spec(A) \subseteq Z$ which is open affine,
we may consider the nilpotent elements of the ring $\eta = \sqrt{(0)}$.
This ideal must be prime as $ab \in \eta$ means $U = V(a) \cup V(b)$,
and so one of $a$ or $b$ is an element of every prime ideal of $A$ by irreducibility.
By definition, the set $\{ \eta \}$ is dense in $U$.
Going one step further, since we know $U \cap V \neq \emptyset$ for all open subsets $V \subseteq X$,
it follows that $\{ \eta \}$ is dense in $X$ as well. 
This point is also unique since being dense in $U$ necessitates being the radical of $A$.

Resuming our discussion on integral schemes,
another deduction we can make is that $X$ is \textit{reduced}.
That is, for any $P \in X$, $\cO_{X,P}$ has no nilpotent elements.
By taking an affine neighbourhood of any point, we can immediately see this holds.
Additionally, when $X$ is both reduced and irreducible,
we notice that the generic point corresponds to just the trivial ideal of each open affine subset,
or else we would be able to find some $P \in X$ for which $\cO_{X, P}$ has nilpotent elements.
As this holds over any affine open subscheme of $X$, for any $U \subseteq X$,
$fg = 0$ over $\cO_X(U)$ grants the same relationship on every affine subset of $U$.
Therefore, by partitioning our open cover $\cV$ of open affine subsets of $V$, we find
\[
    \left(\bigcup_{\substack{V \in \cV \\ f|_V = 0}} V \right)
    \cap \left(\bigcup_{\substack{V \in \cV \\ f|_V \neq 0}} V \right)
    = \emptyset
\]
and so it follows by irreducibility one of the two must be the empty set and the other all of $U$. 
When the condition $f|_V = 0$ holds for all $V \in \cV$, then $f = 0$ and we are done.
Otherwise, since $V \in \cV$ is integral, $g|_V = 0$ for every open affine subset,
and then $g = 0$ as desired.

We summarize this discussion in the following proposition.

\begin{proposition}
    A scheme $X$ is integral if and only $X$ is both reduced and irreducible.
    Moreover, an affine scheme is integral if and only if the defining ring is an integral domain.
\end{proposition}
\begin{proof}
    See previous discussion above, 
    as well as Proposition II.3.1 of \cite{Hartshorne_2013}
\end{proof}

One last aspect of integrality we will enjoy is the notion of a fraction field for the entire scheme.
Indeed, taking an affine subset $U \cong \Spec A$ of an integral scheme $X$, 
which is necessarily an integral domain,
we may localize at the prime ideal $(0)$ and obtain a field of fractions.
As before, this ideal corresponds to the generic point $\eta \in X$,
we obtain the same field $\cO_{X, \eta}$ regardless of choice of affine subset.

Next, let's consider the property of being noetherian,
which applies both to spaces and to rings.
For our space $X$, 
a \textit{noetherian topological space} is defined by the descending chain condition.
That is, for any family closed subsets $(Y_n)_{n \geq 1}$,
satisfying the following descending chain condition,
\[
    Y_1 \supseteq Y_2 \supseteq \cdots,
\]
then it must be the case that there is some sufficiently large $N$ for which $Y_N = Y_n$ for all $n \geq N$ thereafter.
An interesting consequence is that any closed subset of a noetherian topological space can be covered uniquely by a finite number of irreducible closed subsets.

We may also define the \textit{dimension} of $X$ as a topological space as the supremum of such chains with distinct closed sets strictly contained in $X$  by
\[
    \dim X = \sup \{ n : X \supsetneq Y_1 \supsetneq \cdots \supsetneq Y_n \}.
\]
If we fix some closed subset $Z \subseteq X$,
then the \textit{codimension} is defined by the supremum of lengths of descending chains strictly contained in $X$, ending with $Z$,
and no two closed subsets are equal.
This definition can be extended to any other subset $Y \subseteq X$ by considering the infimum of $\codim(Z, X)$ for all closed subsets $Z \subseteq Y$.

For our rings, a \textit{noetherian ring} is one for which all ideals are finitely generated.
Equivalently, it is a ring that satisfies an ascending chain condition (section 1.4 of \cite{Eisenbud_2013}).
Specifically, for a noetherian ring $A$,
that for a family of ideals $(I_n)_{n \geq 1}$ such that
\[
    I_1 \subseteq I_2 \subseteq I_3 \subseteq \cdots,
\]
then it must be the case that there is some sufficiently large $N$ for which $I_N = I_n$ for all $n \geq N$ thereafter.

For any prime ideal $P \in \Spec(A)$, 
we may take its \textit{height} as the supremum of the lengths of chains strictly contained within $P$.
We also define the \textit{Krull dimension} of $A$ by the supremum of all heights of prime ideals.

With so many similarities,
it does not come across as a surprise that these notions coincide.
We summarize the relationships between our terminology in the following proposition.

\begin{proposition}
    Let $X = \Spec A$ be an affine scheme.
    \begin{enumerate}
        \item The dimension of $X$ as a topological space is the same as the Krull dimension of $A$.
        \item $X$ is noetherian if and only if $A$ is noetherian.
        \item If we suppose further that $A$ is an integral domain, 
        which is also a finitely generated $k$-algebra,
        then for any closed irreducible subset $V(P) \cong \Spec(A/P)$,
        \[
            \dim V(P) + \codim(X, V(P)) 
            = \dim(A/P) + {\rm ht}(P) 
            = \dim X,
        \]
        where the two sums are equal term wise.
    \end{enumerate}
\end{proposition}
\begin{proof}
    We refer to Propostion II.3.2 of \cite{Hartshorne_2013},
    Corollary 13.4 of \cite{Eisenbud_2013},
    and our characterization of the irreducible closed subsets of an affine space as the closures of prime ideals.
\end{proof}

To explain some further properties, 
we will examine relative schemes and relatively valued points.

\begin{definition}
    A scheme $X$ is said to be a \textit{scheme over $Y$} when there exists a morphism $X \to Y$.
    We will often write $X/Y$ to denote this,
    or $X/A$ when $Y = \Spec(A)$ for some ring $A$.
\end{definition}

This generally captures the notion that the defining equations and algebras of our scheme are drawn from the rings associated to $Y$.
Notably, since there is a canonical map $\Z \to A$ for any ring $A$,
any affine scheme is over $\Z$ and thus all schemes when gluing is accounted for.
This follows from exercise 7.3.G of \cite{Vakil_2022}.

\begin{definition}
    Given schemes $X$ and $Z$,
    a \textit{$Z$-valued} point on $X$ is a morphism $Z \to X$.
    The space of $Z$-valued points on $X$ is denoted $X(Z)$.
    Moreover, when $Z = \Spec(k)$ for a field $k$,
    we refer these points as \textit{$k$-rational}
    (or rational when $k = \Q$),
    and write $X(k)$.
\end{definition}

To make sense of this,
for a $k$-rational point $f : \Spec(k) \to X$,
the unique prime ideal of $\Spec(k)$ is sent to some point $P \in X$,
so we would like to think of $P$ as the rational point itself.
However, the image in $X$ is not enough to characterize our rational point,
as we also have an associated map of sheaves $\cO_X \to f_* \cO_k$.

To understand this sheaf map, for any $U \subseteq X$,
it is clear that $f_*\cO_k(U) = k$ when $P \in U$,
and $f_*\cO_k(U) = 0$ otherwise.
Moreover, this map is characterized by the map $\cO_{X,P} \to k$ since sheaf maps commute with restriction homomorphisms.

To further dive into this map on the local ring $\cO_{X, P}$,
we recall that the local ring $\cO_{X, P}$ has a unique maximal ideal denoted $\Fm_P$.
And so we may define the \textit{residue field} $\kappa(P) = \cO_{X,P} / \Fm_P$.
Since we require this to be a local homomorphism,
it should be immediate that the kernel of $\cO_{X, P} \to k$ is exactly $\Fm_P$,
and so the map $\kappa(P) \to k$ is an inclusion.
This therefore characterizes our rational points.

Examining the residue field for a given point $P \in X$,
consider the $\cO_{X,P}$ module $\Fm_P / \Fm_P^2$.
As the elements of $\Fm_P$ vanish under multiplication,
we in-fact have a well-defined $\kappa(P)$-vector space.
For the affine variety $V(f_1, \ldots, f_n) \subseteq \A_k^n$,
notice that there is a correspondence between the Jacobian $(\partial f_i / \partial x_j)_{i,j}$
and the vector space $\Fm_P / \Fm_P^2$.


\begin{definition}
    Given a connected scheme $X$ and a point $P \in X$,
    we say that $P$ is \textit{regular} if
    \[
        \dim_{\kappa(P)} \Fm_P / \Fm_P^2 = \dim X.
    \]
    If all the points of $X$ are regular, 
    we that $X$ is a \textit{regular scheme}, or $X$ is \textit{non-singular}.
\end{definition}

For one last construction essential to our geometry, we come to the product of schemes.
Consider for example some ring homomorphisms $f : C \to A$ and $g : C \to A$ for some rings $A$, $B$, and $C$.
With these, note that we may also define morphisms $A \to A \otimes_C B$ and $B \to A \otimes_C B$ by
\[
    a \in A \mapsto a \otimes 1 \qquad 
    b \in B \mapsto 1 \otimes b.
\]
With these ring maps, 
suppose for a fourth ring $R$ that we have homomorphisms $\alpha : A \to R$ and $\beta : A \to R$ for which $\alpha \circ f = \beta \circ g$ as maps $C \to R$.
Then there is a ring homomorphism $A \otimes_C B \to R$ given by
\[
    a \otimes b \mapsto \alpha(a)\beta(b),
\]
which is well-defined by our requirements on $\alpha$ and $\beta$,
such that the maps $\alpha$ and $\beta$ may be factored through $A \otimes_C B$.
Moreover, it is clearly unique since the images of $a \otimes 1$ and $1 \otimes b$ for $a \in A$ and $b \in B$ determine the map on the rest of the ring.
With this in mind,
note that the same must be true for the associated affine schemes with the arrows reversed.
More generally, by taking affine covers,
we may construct a product over general schemes.

\begin{theorem}
    Let $X$, $Y$, and $Z$ be schemes with morphisms $A \to Z$ and $Y \to Z$.
    Then there exists a scheme $X \times_Z Y$,
    referred to as the \textit{fibred product},
    such that there are \textit{projection maps} $\pi_1 : X \times_Z Y \to X$ and $\pi_2 : X \times_Z Y \to Y$ such that,
    for any scheme $W$ with morphisms to $A$ and $B$,
    there is a morphism $W \to X \times_Z Y$ such that the following diagram commutes
    \[
        \begin{tikzcd}
            W \arrow[drr, dashed] \arrow[ddr] \arrow[ddrrr, bend left=50] & & & \\
            & & X \times_Y Z \arrow[dl] \arrow[dr]  & \\
            & X \arrow[dr] & & Y \arrow[dl] \\
            & & Z & 
        \end{tikzcd}
    \]
    The projection morphisms also satisfy that the preimage $\pi_1^{-}(U)$ for an open subset $U \subseteq A$ is the product $U \times_Z B$. 
    Moreover, if $U \subseteq Z$ is affine,
    with $V_1 \subseteq X$ and $V_2 \subseteq Y$ affine contained in preimages of $U$,
    then $V_1 \times_U \times V_2$ is an open affine subset of $X \times_Z Y$ given by the ring tensor.
\end{theorem}
\begin{proof}
    See Theorem II.3.3 of \cite{Hartshorne_2013}.
\end{proof}

With fiber products, we uncover our last general properties of schemes which helps us describe varieties as schemes.

\begin{definition}
    A map $X \to Y$ is said to be \textit{separated} if the diagonal map $\Delta : X \to X \times_Y X$,
    which is derived from the identity $X \to X$ and the morphism $X \to Y$,
    is a closed immersion.
    If the canonical map $X \to \Z$ is separated, we say that $X$ is separated.
\end{definition}

\begin{definition}
    Let $X$ and $Y$ be schemes.
    Defining \textit{projective $n$-space} over $Y$ as $\bbP^n_\Z \times_{\Z} Y$,
    we say that a map $f : X \to Y$ is \textit{projective} if it is factored by a closed immersion $X \to \bbP^n_Y$.
    More generally, the map $f : X \to Y$ is \textit{quasi-projective} if we may factor first by some open immersion into a projective map.
    If $A$ is a ring and $f : X \to \Spec(A)$ is a (quasi) projective map,
    we say that $X$ is (quasi) projective over $A$.
\end{definition}
 
Finally, we come to our very nice schemes referred to as \textit{varieties}.
These are exactly the quasi-projective integral schemes over $k$,
where $k$ is an algebraically closed field.