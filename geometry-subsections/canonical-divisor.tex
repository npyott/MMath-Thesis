To tie our geometry together,
we take a page from differential geometry and consider the derivatives of our functions.
Following from the operations familiar to us for derivatives in ordinary calculus,
let's examine the generalization of K\"{a}hler differentials.

\begin{definition}[Relative Derivation]
    Let $A$ be a commutative ring, 
    $B$ and $A$-algebra,
    and $M$ some $B$-module.
    An \textit{$A$-derivation} $d : B \to M$ is  a map satisfying for all $b_1, b_2 \in B$ and $a \in A$:
    \begin{enumerate}
        \item (Additivity) $ d(b_1 + b_2) = d(b_1) + d(b_2)$.
        \item (Leibniz Rule) $d(b_1b_2) = b_1d(b_2) + b_2d(b_1)$.
        \item (Constant Rule) $d(a) = 0$.
    \end{enumerate}
\end{definition}

Note that the Leibniz rule, constant rule, and additivity pair together to mean that a $k$-derivation over a $k$-vector space is indeed linear.

Paired with our relative derivations,
we also have an associated module which satisfies a certain universal property.
The most straightforward way to go about constructing such a module is by taking
\[
    M = \frac{
        \langle
            db 
        \rangle_{b \in B}
    }{
        \langle 
            d(b_1 + db_2) - db_1 - db_2,
            d(b_1b_2) - b_1db_2 - b_2db_1,
            da
        \rangle_{b_1, b_2 \in B, a \in A}
    },
\]
where we imply the free $B$-modules generated by such symbols.
However, another such way we may do so is by considering the ring $B \otimes_A B$.

Following a construction familiar if you have worked with differential forms, 
suppose we have the map $D : B \otimes_A B \to B$ given on the pure tensors as $D(b_1 \otimes b_2) = b_1b_2$.
Using the kernel $I = \ker D$,
we may determine an $A$-derivation $d : B \to I/I^2$ by
\[
    db = b \otimes 1 - 1 \otimes b + I^2,
\]
where $I/I^2$ inherits $B$-multiplication from $B \otimes_A B$ defined by $b_1(b_2 \otimes b_3) = b_1b_2 \otimes b_3$.
Clearly, without even considering $B$-multiplication,
we immediately see this derivation satisfies additivity and the constant rule.
Additionally, we do also find for $b_1, b_2 \in B$ that
\begin{align}
    d(b_1b_2) & = b_1b_2 \otimes 1 - 1 \otimes b_1b_2 + I^2, \\
    b_1db_2 + b_2db_1 
    & = b_1b_2 \otimes 1 - b_1 \otimes b_2
        + b_2b_1 \otimes 1 - b_2 \otimes b_1 + I^2, \\
    db_1 \cdot db_2 
    & = b_1b_2 \otimes 1 - b_2 \otimes b_1
        - b_1 \otimes b_2 + 1 \otimes b_1b_2 + I^2.
\end{align}
However, since the representative for $db_1 \cdot db_2$ is necessarily an element of $I^2$, it follows that
\[
    b_1db_2 + b_2db_1 = d(b_1b_2) + db_1 \cdot db_2 = d(b_1b_2).
\]
This discussion is summarized in the following proposition.

% Hartshorne or Matsumura?
\begin{proposition}
    Let $B$ be an $A$-algebra,
    both commutative rings.
    There exists a \textit{module of relative differential forms}, 
    denoted $\Omega_{B/A}$ with an $A$-derivation $d : B \to \Omega_{B/A}$ and unique up to isomorphism,
    such that if $d_2 : B \to M$ is another $A$-derivation,
    there there exists a unique map $\varphi : \Omega_{B/A} \to M$ such that the following diagram commutes.
    \[
        \begin{tikzcd}
            B \arrow[d, "d"] \arrow[r, "d_2"] & M \\
            \Omega_{B/A} \arrow[ur, dashed, "\varphi"]
        \end{tikzcd}
    \]
\end{proposition}

\begin{proof}
    See Proposition 26.1 of \cite{Matsumura_1970}.
\end{proof}

\begin{example}
    Let $A = k$ and $B = k[x_1, \ldots, x_n]$ for a number field $k$.
    Then $\Omega_{B/k} = \Span_k \{dx_1, \ldots, dx_n\}$,
    and the derivation map $d : B \to \Omega_{B/k}$ corresponds to the derivative of a given polynomial as a complex function.
\end{example}

Let's now construct a sheaf which corresponds to our module of relative differentials.
Suppose that $X$ and $Y$ are schemes and we are given a map $f : X \to Y$.
Consider first the case where $X = \Spec(B)$ and $Y = \Spec(A)$.
We recall that $X \times_Y X \cong \Spec(B \otimes_A B)$,
and there is a map $X \to X \times_Y X$ which comes from the map $D : B \otimes_A B \to B$ as defined above.
As this ring map is surjective,
we have identified $X$ as a closed subscheme of $X \times_Y X$ whose ideal sheaf is generated by the kernel of $D$.

With this setup,
we may now impose that $\Omega_{X/Y}$ be defined as $(\Omega_{B/A})^\sim$,
where we now obtain a derivation $d : \cO_X \to \Omega_{X/Y}$.
It should be noted that the localization of a module of relative differential forms is precisely the differential forms of the localized ring,
so our derivation map remains well defined as a sheaf map.
For more general schemes $X$ and $Y$,
it suffices to patch both with affine open subsets and glue these modules together to define the sheaf $\Omega_{X/Y}$.

\begin{example}
    Let $X = \bbP^1_k$ with projective coordinates $(x : y)$,
    and take $Y = \Spec(k)$ for an algebraically closed field $k$.
    We may compute $\Omega_{\bbP^1_k/k}$ as isomorphic to the sheaf $\cO(-2)$.
    To see why, fix an affine patch $U = D_+(x)$.
    On this patch, we know that $\Omega_{U/k}$ is a free sheaf of rank 1 generated by the global section $d\left( \frac{y}{x} \right)$.
    By using a similar argument on $V = D_+(y)$, 
    we have on the open cover $\bbP^1_k = U \cup V$
    \[
        \Omega_{\bbP^1_k/k}|_{U} \cong d\left( \frac{y}{x} \right) \cO_{\bbP^1_k}|_{U} \qquad
        \Omega_{\bbP^1_k/k}|_{V} \cong d\left( \frac{x}{y} \right) \cO_{\bbP^1_k}|_{V}.
    \]
    On the intersection $U \cap V$, it follows from our rules of derivation (or by standard calculus rules) that
    \[
        \frac{x}{y} \cdot d\left( \frac{y}{x} \right)
        = - \frac{y}{x} \cdot d\left( \frac{x}{y} \right).
    \]
    Therefore, by comparing local generators,
    $\Omega_{\bbP^1_k/k}$ is isomorphic to the sheaf generated by the Cartier divisor $\{(U, \frac{x}{y}), (V, \frac{y}{x})\}$.
    Since this divisor has a pole at $\{y = 0\}$ and a pole at $\{x = 0\}$,
    and no zeroes or poles anywhere else,
    it must be isomorphic to $\cO(-2)$.
\end{example}

An interesting fact of the sheaf of relative differentials is that it encodes the information regarding whether a particular scheme is non-singular by its rank should it be a free sheaf.
Unfortunately, this also means that for surfaces and beyond,
we no longer have an invertible sheaf and cannot connect it to the Picard group.
However, we may recover this via the exterior algebra.

\begin{definition}
    Let $X/k$ be non-singular.
    The \textit{canonical divisor sheaf} on $X$, is $\omega_X = \bigwedge^n \Omega_{X/k}$,
    where $\bigwedge^n \Omega_{X/k}$ is the sheaf associated to the presheaf
    \[
        U \mapsto \bigwedge^n \Omega_{X/k}(U).
    \]
    We often write $K_X$ as the canonical divisor associated to $\omega_{X/k}$ when we have all divisor groups isomorphic.
\end{definition}

We refer to such a sheaf as canonical since it has been defined without any choices.
Notably, each of our previous divisors which granted us maps to projective space each came with some choice of coordinates or hyperplanes.
Now, however, we have a purely geometric way to examine non-singular schemes.
Let's investigate a few propositions which help give a sense for how this sheaf behaves.

\begin{proposition}
    Let $X = \bbP^n$.
    Then $\omega_X \cong \cO(-n - 1)$.
\end{proposition}

\begin{proof}
    While a similar argument follows for $n \geq 2$ as it did for $\bbP^1$,
    this also follows from Theorem II.8.13 of \cite{Hartshorne_2013}.
\end{proof}

We also recall Exercise II.8.3 and Exercise II.8.5 of \cite{Hartshorne_2013} in the following two propositions.

\begin{proposition}
    Let $X$ and $Y$ be non-singular schemes over $k$.
    Then if $\pi_1 : X \times_k Y \to X$ and $\pi_2 : X \times_k Y \to Y$ are the projection maps,
    we have
    \[
        K_{X \times_k Y}
        \sim \pi_1^* K_X + \pi_2^* K_Y.
    \]
\end{proposition}

\begin{proposition}
    Let $X$ be a non-singular scheme,
    with $Y \subseteq X$ a non-singular closed subscheme with $\codim(Y, X) = r \geq 2$.
    If $\pi : \wt X \to X$ is the blow up of $X$ along $Y$,
    with exceptional divisor $E$.
    Then
    \[
        K_{\wt X} = \pi^* K_X + (r - 1)E.
    \]
\end{proposition}