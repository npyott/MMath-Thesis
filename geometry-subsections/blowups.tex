One of the advantages of algebraic geometry is the ability to analyze a space regardless of coordinates or embedding. 
As such, we often look towards birational maps which capture the geometry of a given space but may transform into an easier to understand or better behaved space.
One such way of constructing these birational maps comes from the idea of a blow up.
We capture blow ups unique up to isomorphism using the following universal property definition.

\begin{definition}[Blow up]
    Let $Y \subseteq X$ be a closed subscheme.
    The \textit{blowup of $X$ along $Y$}, with \textit{centre} $Y$, 
    is a space $\wt{X}$ and a map $\pi : \wt X \to X$ which is an isomorphism on $X \sm Y$.
    We denote $E = \pi^{-1}(Y)$ as the \textit{exceptional divisor}, which is an effective Cartier divisor.

    Finally, for any space $W$ with closed subscheme $Z$ for which the ideal sheaf is locally invertible,
    along with a map $\varphi : W \to X$ satisfying $Z =\varphi^{-1}(Y)$,
    then there exists a unique map $\psi : W \to \wt X$ which satisfies $\pi \circ \psi = \varphi$.
    This is captured by the following commutative diagram.

    \[
        \begin{tikzcd}
            Z \arrow[r, hook] \arrow[ddr] \arrow[drr, dashed] & W \arrow[ddr] \arrow[drr, dashed] & \\
            & & E \arrow[r, hook] \arrow[dl] & \wt X \arrow[dl] \\
            & Y \arrow[r, hook] & X &
        \end{tikzcd}
    \]
\end{definition}

Let's investigate this definition by examining the example on the affine case ${X = \Spec(A)}$,
where our closed subscheme $Y$ is described by an ideal $I = \langle f_1, \ldots, f_r \rangle$. 
We claim that a good candidate for $\wt X$ is $\Proj(\bigoplus_{d \geq 0} I^dt^d)$,
where $I^d = A$ and the parameter $t$ is kept to account for the degree in our graded ring.
We may write $A[It]$ as a shorthand for the graded module in question.

A natural place to start is by first establishing a map $\pi : \wt X \to X$.
In this case, we have a very simple and natural map by taking $P \in \Proj(A[It])$ and considering $P \cap A$,
which simply follows from the inclusion $A \hookrightarrow A[It]$.
Under such a mapping, let's consider our exceptional divisor $E = \pi^{-1}(Y)$.
Following the inclusion $Y \hookrightarrow X$ by prime ideals containing $I$,
we have the equivalence below.
\begin{align*}
    P = \bigoplus_{d \geq 0} P_d t^d \in \pi^{-1}(Y) 
        & \iff I \subseteq \pi(P) = P_0 \\
        & \iff \forall d \geq 0, I^{d + 1}t^d = I \cdot I^dt^d \subseteq P_0 \cdot I^dt^d \subseteq P_dt^d. 
\end{align*}
Therefore, as $P \in E$ contains $I^{d + 1}t^d$ for the degree $d$ components of the homogeneous prime ideal $P$,
our exceptional divisor $E$ may be thought of as cut out by the ideal sheaf $\cI_E = (\bigoplus_{d \geq 0} I^{d + 1}t^d)^{\sim}$.

We check as well that this ideal sheaf is indeed invertible.
To do so, consider for some $1 \leq j \leq r$ the section $f_j t \in A[It](1)$.
On the open set $D_+(f_jt) \cong \Spec(A[It])_{(f_jt)}$, note we have
\[
    A[It]_{(f_jt)}
        = A + \frac{1}{f_j}I + \frac{1}{f_j^2}I^2 + \cdots
\]
Note as well that since $f_j^k \in I^k$ for all $k \geq 1$, 
it follows that we may cut off arbitrarily many terms from the start as $A \subseteq \frac{1}{f_j}I \subseteq \frac{1}{f_j^2}I^2$ and so forth.
Likewise, taking the ideal sheaf associated to $E$ and localizing to $D_+(f_jt)$,
we have from definition that
\[
    \cI_E|_{D_+(f_jt)}
    = \left(\bigoplus_{d \geq 0} I^{d + 1}t^d\right)_{(f_jt)}
    = I + \frac{1}{f_j}I^2 + \frac{1}{f_j^2}I^3 \cdots
\]
demonstrating that $\cI_E|_{D_+(f_j t)} \cong f_j \cO_{\wt X}|_{D_+(f_j t)}$.
Note as well the parallels of $\cI_E$ to $\cO_{\wt X}(1)$ as they are the same invertible sheaf.

Next, consider a pair $Z \hookrightarrow W$ with $\cI_Z$ invertible,
and a map $\varphi : W \to X$ such that $Z = \varphi^{-1}(Y)$.
With our map $\varphi$, it is easy to construct a map $\psi : W \to \wt X$ which factors through $\pi : \wt X \to X$ as for any $Q \in W$,
\[
    \psi(Q)= \bigoplus_{d \geq 0} (\varphi(Q) \cap I^d) t^d.
\]
A full verification of our universal property follows from Proposition II.7.14 \cite{Hartshorne_2013},
and shows we have indeed constructed the blow up $\pi : \wt X \to X$.

In fact, we have indeed shown that the blow up exists in general.
We explore this and some consequences in the following propositions and corollaries.

\begin{proposition}
    Blow ups are locally defined. 
    That is, given $U \subseteq X$ and a blow up $\pi : \wt X \to X$ with centre $Y$ and exceptional divisor $E$,
    $\pi^{-1}(U)$ is the blow up of $U$ along $Y \cap U$ with exceptional divisor $E\cap \pi^{-1}(U)$.
\end{proposition}

\begin{proof}
    To show this, we need only demonstrate the universal property.
    For this, consider $Z \hookrightarrow W$ satisfying all necessary conditions. 
    We may immediately draw the following diagram.
    \[
        \begin{tikzcd}
            Z \arrow[r, hook] \arrow[d] & W \arrow[d] & E \cap \pi^{-1}(U) \arrow[r, hook] \arrow[d] \arrow[dll] & \pi^{-1}(U) \arrow[d] \arrow[dll] \\
            Y \cap U \arrow[r, hook] \arrow[d] & U \arrow[d] & E \arrow[r, hook] \arrow[dll] & \wt X \arrow[dll] \\
            Y & X
        \end{tikzcd}
    \]
    Following our maps $Z \to Y$ and $W \to X$, 
    we may use our universal property to determine a map $W \to \wt X$.
    However, since it must commute with the map $W \to X$ which has image contained in $U$,
    we see that we in-fact have a map $W \to \pi^{-1}(U)$,
    which satisfies all requirements.
\end{proof}

\begin{corollary}
    Let $\pi: \wt X \to X$ be a blow up along $Y$ with exceptional divisor $E$.
    Then $\pi : \wt X \sm E \to X \sm Y$ is an isomorphism,
    and hence $\pi$ is birational.
\end{corollary}

\begin{proof}
    For this, we simply consider that $\pi^{-1}(X \sm Y)$ is the blow up of $X \sm Y$ along the empty set.
    However, the empty set is already cut out by the ideal sheaf generated by $1$ on any open subset. 
    Therefore $X \sm Y$ satisfies the universal property and hence is isomorphic to $\pi^{-1}(X \sm Y) = \wt X \sm E$.
\end{proof}

We also cite Exercise 23.2.A of \cite{Vakil_2022} in the following proposition.

\begin{proposition}
    Blow ups may be glued together.
    That is, if $\cup_{i \in I} U_i$ is an open cover of $X$,
    and there exists blow ups $\pi_i : \wt U_i \to U_i$ along center $Y \cap U_i$,
    then there exists a blow up $\pi : \wt X \to X$ along center $Y$.
\end{proposition}

With the basics of construction covered, 
let's examine some notable properties of blow ups.
For our first example, let's see how blow ups separate crossing lines and hence could resolve singularities, 
though we will not dive further into this as it is unnecessary for our cases.
We first take a quick detour to define the \textit{strict transform}.

\begin{lemma}
    Let $Y, Z \subseteq X$ be closed subschemes.
    If $\wt Z$ is the blow up of $Z$ along $Y \cap Z$ and $\wt X$ is the blow up of $X$ along $Y$,
    then there exists a closed immersion $\wt Z \hookrightarrow \wt X$.
    Moreover, we may draw the following commutative diagram.
    \[
        \begin{tikzcd}
            \wt Z \arrow[r, hook] \arrow[d] & \wt X \arrow[d] \\
            Z \arrow[r, hook] & X
        \end{tikzcd}
    \]
\end{lemma}

\begin{proof}
    See Corollary II.7.15 of \cite{Hartshorne_2013}.
\end{proof}

\begin{definition}[strict/proper transform]
    Taking $X, Y, Z$ and $\wt Z$ as above, we refer to  $\wt Z$ as the \textit{strict transform} of $Z$.
\end{definition}

\begin{proposition}
    Let $Y \subseteq X$ be the intersection $Y = \cap_{i = 1}^r Y_i$ such that $\cI_Y = \sum_{i = 1}^r \cI_{Y_i}$.
    Then if $\pi: \wt X \to X$ is the blow up of $X$ along $Y$,
    then the intersection of the strict transforms of $Y_i$ is empty in $\wt X$.
\end{proposition}

\begin{proof}
    To begin, we note that this suffices to prove locally since any intersection would lie in the preimage of some open subset of $X$.
    Therefore, we may assume $X = \Spec(A)$, $Y = \Spec(A / I)$ for some ideal $I \subseteq A$, and $Y_i = \Spec(A / J_i)$ for some ideal $J_i \subseteq A$ for each $1 \leq i \leq r$.
    Next, note that $Y \cap Y_i$ is then given as $V(I/J_i) \subseteq Y_i$,
    so we may write each strict transform and our blow up $\wt X$ as
    \[
        \wt Y_i = \Proj\left(
            \bigoplus_{d \geq 0} (I/J_i)^d t^d
        \right), \qquad
        \wt X = \Proj\left(
            \bigoplus_{d \geq 0} I^d t^d
        \right).
    \]
    With this, we may proceed by noting that there is a clear surjective graded ring homomorphism $A[It] \to A[(I/J_i) t]$ which gives each closed immersion.
    Therefore, any prime ideal in the image of a given closed immersion must contain the kernel of this graded ring homomorphism.
    Hence, if $P \in \cap_{i = 1}^r \wt Y_i$,
    then by examining the degree one terms in each kernel,
    it follows that $J_i t \subseteq P$ for each $1 \leq i \leq r$.
    However, since $I = \sum_{i = 1}^r J_i$ by hypothesis,
    we must have $I t \subseteq P$ and so $P$ contains all of $A[It]_+$,
    which is a contradiction.
\end{proof}

Another useful way strict transforms and blow ups come together is in the pullback of an effective Cartier divisor.
Let $\pi : \wt X \to X$ be a blow up of a non-singular variety with irreducible, non-singular center $Y$ of codimension at least 2 and exceptional divisor $E$.
We will also assume that $\wt X$ is non-singular as well.
The first objective is to determine the relation between $\Pic(\wt X)$ and $\Pic(X)$.

To explore this further, recall that we have the exact sequence
\[
    \Z \to \Pic(\wt X) \to \Pic(\wt X \sm E) \to 0.
\]
In fact, this is a short exact sequence.
Consider the fact that $\pi^* : K(X) \to K(\wt X)$ is an isomorphism as it is the pullback of a birational map.
Thus, if $mE = \rmDiv(f)$ for some $m \in \Z$ and $f \in K(\wt X)$,
then we may write $mE = \pi^*\rmDiv(g) = \rmDiv(g \circ \pi)$ for some $g \in K(X)$.
However, this implies that $g$ may not vanish outside $Y$,
but because $\codim(Y, X) \geq 2$,
this is only possible if $g$ is a non-vanishing constant and $m = 0$.

Next, we also have isomorphisms between $\Pic(\wt X \sm E)$, $\Pic(X \sm Y)$, and $\Pic(X)$,
where the last isomorphism is due to the fact $\codim(Y, X) \geq 2$.
Therefore, we may now write a short exact sequence as
\[
    0 \to \Z \to \Pic(\wt X) \to \Pic(X) \to 0
\]
Lastly, we also have a section $\pi^* : \Pic(X) \to \Pic(\wt X)$ which composes with the previous morphism as the identity.
To confirm this, fix some Cartier divisor $D = \{(U_i, f_i)\}_{i = 1}^r \in \Pic(X)$.
We then find that
\begin{align*}
    (\pi^* D)|_{\pi^{-1}(X \sm Y)}
    & = \{(\pi^{-1}(U_i), f_i \circ \pi)\}|_{\wt X \sm E} \\
    & = \{
            (
                \pi^{-1}(U_i) \cap \wt X \sm E, 
                f_i \circ \pi|_{\wt X \sm E}
            ) 
        \} \\
    & = \{
            (
                \pi|_{\wt X \sm E}^{-1}(U_i \cap X \sm Y),
                f_i|_{X \sm Y} \circ \pi|_{\wt X \sm E}
            )
        \} \\
    & = \pi|_{\wt X \sm E}^* D|_{X \sm Y}.
\end{align*}
As the map $\Pic(\wt X) \to \Pic(X)$ follows by restricting to $\Pic(\wt X \sm E)$,
and then following $\pi|_{\wt X \sm E}*$ to $\Pic(X \sm Y)$,
we see that this composition is indeed the identity on $\Pic(X)$.
Therefore, we may write $\Pic(\wt X) \cong \Pic(X) \oplus \Z$.

Since an effective Cartier divisor $D$ on $X$ corresponds to ideal sheaf of a closed subscheme via $\cL(-D)$, 
it is natural for us to associate a strict transform $\wt D$ to this divisor.
As such, locally writing out $D = \{(U_i, f_i)\}_{i = 1}^r$,
we also consider the pullback
\[
    \pi^*D = \{(\pi^{-1}(U_i), f \circ \pi)\}_{i = 1}^r.
\]
In the case where $Y$ is a point, 
we find that $\pi^*D = \wt D + \ord_Y(D) E$, where $E = \pi^{-1}(Y)$ is the exceptional divisor,
following from Exercise 23.4.P of \cite{Vakil_2022}. 
To clarify, $\ord_Y(D) \geq 0$ is defined such that if $\eta \in Y$ is the generic point,
and $D|_U = \rmDiv(f)$ for some neighbourhood $U \subseteq X$ of $\eta$,
then $f \in \Fm_\eta^{\ord_Y(D)} \sm \Fm_\eta^{\ord_Y(D) + 1}$.
This is well-defined regardless of neighbourhood for the same reason the usual mapping from Cartier divisors to Weil divisors is.
 
