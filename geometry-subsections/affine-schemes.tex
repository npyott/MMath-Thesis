Next, let's create a topological space for which a given ring will act like a space of functions.
A natural place where functions and rings coincide are polynomial rings,
and indeed this is a motivating example and basis for all of algebraic geometry with the so-called affine spaces.

Consider the polynomial ring $A = k[x_1, \ldots, x_n]$ for some algebraically closed field $k$ and integer $n \geq 1$.
Common sets of interest include the zero sets of some polynomials $f_1, \ldots, f_m \in A$.
However, to speak of a zero set, it is not always clear what exactly this entails for general rings where the notion of ``plugging-in" is not well-defined.
For exactly these cases, it is convenient to note that a point ${(a_1, \ldots, a_n) \in \A_k^n}$,
where $\A_k^n$ can be regarded as $k^n$ for now,
is a common zero of our functions precisely when we have
\[
    f_1 \equiv \cdots \equiv f_m \equiv 0  \mod \langle x_1 - a_1, \ldots, x_n - a_n \rangle.
\]

In fact, even for the non-vanishing points of $\A_k^n$, 
it is easy to see that the value of a given polynomial $f(a_1, \ldots, a_n)$ can be seen to be the unique representative in $k$ modulo the ideal generated by the polynomials $a_1 - x_1, \ldots, a_n - x_n \in A$. 
This can even be generalized to take an integer $n \in \Z$ and use it as a function on the primes of $\Z$ given by
\[
    p \mapsto n \mod p,
\]
and $n$ vanishes precisely at those $p$ which divide it,
or equivalently using ideals, when ${(n) \subseteq (p)}$. 

We are now ready to define a topological space derived from a ring for which the ring itself will provide us a sheaf of functions.

\begin{definition}[Zariski Topology]
    Given a commutative ring $A$, we define the \textit{spectrum of $A$}, written $\Spec(A)$, to be the set of all prime ideals of $A$.
    The \textit{Zariski Topology} will be generated by open subsets given for $f \in A$ of the form
    \[
        D_f = \{ P \in \Spec(A) : f \notin P \},
    \]
    or equivalently, with closed sets given by the ideals $I \subseteq A$ as
    \[
        V(I) = \{ P \in \Spec(A) : I \subseteq P \}.
    \]
\end{definition}

Let's first explore our topology.
Note that the sets $D_f$ do in fact form a basis for a topology.
\begin{enumerate}
    \item Given $P \in \Spec(A)$, 
    we may find some $f \in A \sm P$ so that $P \in D_f$.
    \item For $f, g \in A$, we have $D_f \cap D_g = D_{f g}$ as for any $P \in \Spec(A)$, 
    $f g \in P$ if and only if $f \in P$ or $g \in P$.
\end{enumerate}

Examining our other topology, we see that the sets $V(I)$ do in fact form a system of closed sets since we may likewise check a few basic facts.
\begin{enumerate}
    \item It is clear that $V(\langle 0 \rangle) = \Spec(A)$ and $V(A) = \emptyset$.
    \item Given $I, J \subseteq A$, we find that $V(IJ) = V(I) \cup V(J)$, following from basic facts of prime ideals.
    Note that this implies $V(P)$ is an \textit{irreducible closed set} in the topology as we cannot write $P = IJ$ non-trivially.
    \item For a system of ideals $I_j$ indexed by $j \in S$, $V( \sum_{j \in S} I_j ) = \cap_{j \in S} V(I_j).$
    This one is only slightly less obvious as clearly $\sum_{j \in S} I_j \subseteq P$ implies $V(I_j)$ contains $P$ for all $j \in S$.
    Conversely, if $P \in V(I_j)$ for all $j \in S$, 
    then as the smallest ideal containing each $I_j$, $\sum_{j \in S} I_j \subseteq P$.
\end{enumerate}

We may also explain briefly why the two topologies given agree.
Clearly, we have $D_f = \Spec(A) \sm V((f))$ from reading definitions.
Checking that the basis generates this topology as well, 
we may consider an arbitrary open subset $U = \Spec(A) \sm V(I)$ from a given ideal $I \subseteq A$ and a point $P \in U$. 
We may then take any $f \in I \sm P$ (non-empty or else $P \in V(I)$), such that $P \in D_f \subseteq U$.

In this case, we will refer to the closed sets as \textit{algebraic sets} or \textit{affine varieties}.
Recall that $\Spec(A)$ is defined as the set of all prime ideals while $\A_k^n$ was originally defined by the tuples $k^n$.
A \textit{closed point} is some $P \in \Spec(A)$ such that $\ol{\{P\}} = \{P\}$.
It is clear that the closed points of $\A_k^n$ will then be maximal ideals of $A$, and hence the following correspondence
\[
    (a_1, \ldots, a_n) \in k^n \biject \langle x_1 - a_1, \ldots, x_n - a_n \rangle \in \A_k^n.
\]
For our other prime ideals which are not maximal, these will then correspond to irreducible algebraic sets as mentioned previously, 
such as curves or surfaces contained in $\A_k^n$ defined by algebraic equations.
Under this correspondence, it is clear that our closed sets $V(I) \subseteq \A^n_k$ correspond to the points on which all polynomials in $I$ vanish.

Next, we turn towards Hilbert's Nullstellensatz from Theorem 1.3A of \cite{Hartshorne_2013}. 

\begin{theorem}[Hilbert's Nullstellensatz]
    Consider the polynomial ring $A = k[x_1, \ldots, x_n]$ over an algebraically closed field $k$.
    Let $I \subseteq A$ be an ideal and consider $V(I) \subseteq \A_k^n$.
    If $f \in A$ vanishes along $V(I)$, then $f^k \in I$ for some integer $k \geq 1$. 
\end{theorem}

In this context, if we find $g \in A$ vanishes only on a subset of $V(f)$,
then $f$ vanishes along $V(g)$ so $f^k \in \langle g \rangle$ for some $k \geq 1$. 
Therefore, we may write $f^k = gh$ for some $h \in A$ and
\[
    \frac{1}{g} = \frac{h}{gh} = \frac{h}{f^k}.
\]

Hence, our ring of functions is exactly the functions whose denominators are powers of $f$.
This may be expressed as the localization $A_f$ of $A$ by the multiplicative system $\{ f^n \}_{n \geq 0}$.

This definition works well even on more general rings $A$.
Thinking of restriction maps, consider $D_g \subseteq D_f \subseteq \Spec(A)$ for some $f, g \in A$.
While it is obvious how to restrict $a \in A$ to $\frac{a}{1} \in A_f$,
it is not as clear how to restrict $A_f$ to $A_g$.
Fortunately,
we have the following chain of equivalences
\[
    D_g \subseteq D_f 
    \iff V(f) \subseteq V(g)
    \iff \sqrt{\langle g \rangle} \subseteq \sqrt{\langle f \rangle},
\]
where $\sqrt{I}$ is the radical of the ideal $I \subseteq A$,
and we may use the following characterization from Corollary 2.21 of \cite{Eisenbud_2013},
\[
    \sqrt{I} 
    = \bigcap_{\substack{
        P \in \Spec(A) \\
        I \subseteq P
    }} P
    = \{ a \in A : \exists n \geq 1, a^n \in I \}.
\]
While the second equality is clear from the definition as an intersection of prime ideals,
we also find from the other part of the definition that since $g \in \sqrt{\langle g \rangle}$, 
there must be some $n \geq 1$ such that $g^n \in \langle f \rangle$.
Writing $g^n = fh$, it is clear that $f \in A$ maps to a unit in $A_g$ under the canonical ring homomorphism. 
Thus, we may use the universal property of localization to determine a unique ring homomorphism $A_f \to A_g$ such that the following diagram commutes.
\[
    \begin{tikzcd}
        A \arrow[r] \arrow[d] & A_g \\
        A_f \arrow[ur, dashed]
    \end{tikzcd}
\]

Let's now examine what our expected stalks should be by considering the ring $\Z$.
Taking $n \in \Z$ arbitrarily, we may consider $n^{-1}$ defined as a function on the open set $D_n$ since for any $(p) \in D_n$,
$n$ has a multiplicative inverse in the ring $\Z/p\Z$ by the fact $\gcd(n, p) = 1$.
Thus, if we fix a prime $p$ and examine all open basis sets $D_n$ containing $p$,
we find that the pair $\langle D_n, n^{-1} \rangle$ belongs to our stalk.
After considering all such possibilities in this regard, we should expect our stalk to be
\[
    \Z_{(p)} = \left\{
        \frac{a}{b} : a \in \Z, b \in \Z \setminus (p)
    \right\},
\]
which is the localization of $\Z$ by the multiplicative system of integers not contained in $(p)$.


With this, for an arbitrary ring $A$, we determine our sheaf of functions $\cO_{\Spec(A)}$.

\begin{theorem}
    Given a ring $A$ with $X = \Spec(A)$,
    there exists a unique sheaf of rings $\cO_X$ referred to as the \textit{structure sheaf} such that the following three properties hold.
    \begin{enumerate}
        \item Our global sections are the entire ring, given by $\cO_X(X) \cong A$.
        \item For any $f \in A$, $\cO_X(D_f) \cong A_f$. 
        The restriction map $\cO_X(X) \to \cO_X(D_f)$ is given by the canonical ring homomorphism $A \to A_f$,
        and likewise for $\cO_X(D_f) \to \cO_X(D_g)$ by $A_f \to A_g$ when $D_g \subseteq D_f$ for some $g \in A$.
        \item For any $P \in X$, $\cO_{X, P} \cong A_P$.
    \end{enumerate}
\end{theorem}

\begin{proof}
    Refer to Proposition II.2.2 of \cite{Hartshorne_2013}.
\end{proof}

With the previous theorem from as the defining characteristics of our affine space,
the best way to truly understand affine spaces is via maps between them.
Indeed, there is a very natural way to interpret all the maps $\Spec(A) \to \Spec(B)$,
and this is through ring homomorphisms $B \to A$.
Certainly, by following directly from the definition,
if $\varphi : B \to A$ is a ring homomorphism,
then $\varphi^{-1}(P) \subseteq B$ is a prime ideal whenever $P \subseteq A$ is a prime ideal.
Moreover, for any distinguished open set $D_b \subseteq B$,
it is clear that
\begin{align*}
    P \in (\varphi^{-1})^{-1}(D_b)
    & \iff \varphi^{-1}(P) \in D_b \\
    & \iff b \notin \varphi^{-1}(P) \\
    & \iff \varphi(b) \notin P \\
    & \iff P \in D_{\varphi(b)},
\end{align*}
and hence $\varphi^{-1} : \Spec(A) \to \Spec(B)$ is continuous.

Let's consider a case that will be of interest and how these maps fit together with the geometry of the situation.
Consider an algebraically closed field $k$ and let $A$ and $B$ be finitely generated $k$-algebras with ring homomorphism $\varphi : B \to A$.
Explicitly, for some integers $n, m \geq 1$, we consider surjective ring homomorphisms
\[
    \alpha : k[x_1, \ldots, x_n] \to A, \qquad
    \beta : k[y_1, \ldots, y_m] \to B.
\]
Notice that we may understand the map $\varphi$ entirely from how it acts on the images of $y_1, \ldots, y_m \in B$,
where the quotient by the kernel of $\beta$ is implicit.
That is, for each $1 \leq i \leq m$,
there is some function $f_i(x_1, \ldots, x_n) \in k[x_1, \ldots, x_n]$ for which
\[
    \varphi(y_i) = f_i(x_1, \ldots, x_n),
\]
where again the quotient by the kernel of $\alpha$ is taken implicitly.
Note that $f_i$ may be taken equivalently up to the kernel of $\alpha$ and hence from the ring $A$ under isomorphism,
something we will note shortly.

Staying close to the geometry, let $P \in \Spec A$ be a maximal ideal,
which corresponds to a maximal ideal of $k[x_1, \ldots, x_n]$ containing $\ker \alpha$.
As discussed previously, this maximal ideal is generated by $x_i - a_i$ for $1 \leq i \leq n$.
Therefore, we find that
\[
    \langle y_1 - f_1(a_1, \ldots, a_n), \ldots, y_m - f_m(a_1, \ldots, a_m) \rangle \subseteq \varphi^{-1}(P),
\]
since $f_i(x_1, \ldots, x_n) - f_1(a_1, \ldots, a_n) \in P$.
However, this containment is equality since the ideal on the left is a maximal ideal.

Therefore, we have seen in the fundamental case of maximal ideals of finitely generated $k$-algebras,
that our functions $\Spec(A) \to \Spec(B)$ are actually given by the polynomial ring $A$,
mapping
\[
    (a_1, \ldots, a_n) \in \Spec(A) 
    \mapsto 
    (f_1(a_1, \ldots, a_n), \ldots, f_m(a_1, \ldots, a_n)) \in \Spec(B).
\]

Based on the previous example, we see that there is a good reason to believe that the spectrum of a quotient ring is a subspace.
Consider the ring surjection $A \to A/I$, where $I \subseteq A$ is any ideal of the ring $A$.
In this case, the map $\Spec(A/I) \to \Spec(A)$ is exactly the map which sends an ideal $P/I$ to the ideal $P$,
where $I \subseteq J \subseteq A$.
It is an easy exercise to verify that all prime ideals of $A/I$ are given in this form.

Note that this map is certainly injective as the ideal $P/I$ is generated by the elements of $P$ under the surjection $A \to A/I$,
and hence $\Spec(A/I)$ can be viewed as a subspace of $\Spec(A)$.
Furthermore, this subspace is also a closed set,
as it is exactly $V(I) \subseteq \Spec(A)$.
In this way, we see that we can recover all the closed subspaces of $\Spec(A)$,
and we refer to the map $\Spec(A/I) \hookrightarrow \Spec(A)$ as a \textit{closed immersion}.

Just as for closed sets, we too have inclusion for open sets defined by ring maps.
Let $f \in A$ be arbitrary and consider the localization $A_f$ with the canonical ring homomorphism $\varphi: A \to A_f$,
which is the same as the restriction map $\cO_X(X) \to \cO_X(D_f)$
Since we know $\cO_X(D_f) \cong A_f$, 
we should expect $\Spec(A_f)$ to be bijective correspondence to $D_f \subseteq \Spec(A)$.
Let $P \in \Spec(A_f)$ be a prime ideal.
Immediately, we find $f^n \notin \varphi^{-1}(P)$ for any $n \geq 1$,
or else $P$ generates all of $A_f$ by $\frac{1}{f^n} \cdot \frac{f^n}{1} = \frac{1}{1}$.
From this, we may conclude that not only is $\varphi^{-1}(P) \in D_f$,
but also that the for any element $\frac{a}{f^n} \in A_f$ represented by some $a \in A$ and $n \geq 1$, 
it is equivalent for $\frac{a}{f^n}$ to belong to $P$ as it is for $\frac{a}{1}$ to belong to $P$,
where the latter is also equivalent to the condition that $a$ belongs to $\varphi^{-1}(P)$.

Thus, prime ideals of $A_f$ are generated by elements in the image of the map $A \to A_f$,
and so it is clear that $\Spec(A_f)$ is precisely $D_f$.
The map $\Spec(A_f) \to \Spec(A)$ is therefore able to be seen as the inclusion, 
or specifically the \textit{open immersion},
of $D_f \subseteq \Spec(A)$.
In fact, for any open subset $U \subseteq \Spec(A)$,
we may define the inclusion along $U$ by restricting to distinguished open sets as these maps are certainly compatible on overlap.

Let's look at some further cases as to how these ring maps induce maps between the structure sheaves of our affine spaces.
Let $f : X \to Y$ be given by a ring map $\varphi : B \to A$,
where $X = \Spec(A)$ and $Y = \Spec(B)$.
Take $U \subseteq \Spec(B)$ to be the distinguished open subset $D_b$ for some $b \in B$,
and recall from previous discussion that $f^{-1}(U) = D_{\varphi(b)}$.
What we wish to consider is a map $f^\#(U) : \cO_Y(U) \to f_* \cO_X(U)$,
so that we may ultimately construct a sheaf map $f^\# : \cO_Y \to f_* \cO_X$.

In this case, it is easy since $\cO_Y(U) = B_b$ and $f_* \cO_X(U) = A_{\varphi(b)}$ and we may define the map of rings $B_b \to A_{\varphi(b)}$ as simply
\[
    \frac{s}{b^k} \mapsto \frac{\varphi(s)}{\varphi(b)^k},
\]
which is well-defined and injective when $\ker \varphi = 0$ by checking on equivalent fractions.
Importantly, this definition agrees when you restrict to overlapping distinguished open subsets,
and so we may indeed glue these maps together to form a sheaf map $f^\# : \cO_Y \to f_* \cO_X$.

Lastly, we may consider the stalks.
Let $(f, f^\#) : (X, \cO_X) \to (Y, \cO_Y)$ be as above and fix some prime $P \in \Spec(A)$.
Just as before on distinguished open subsets, 
we obtain a map $f^\#_P : \cO_{Y, f(P)} \to \cO_{X, P}$ by applying $\varphi$ to the numerator and denominator of the fractions in $B_{f(P)}$.
Moreover, this is a \textit{local homomorphism} since $(f^\#_P)^{-1}(P A_P) = f(P) B_{f(P)}$,
sending the unique maximal ideal on one local ring to the unique maximal ideal of the other.
