While schemes and the rings associated to them have proven to be very useful,
there is no reason we cannot examine other sheaves on our topological spaces.
A natural step to obtaining further algebras is to go from sheaves of rings to sheaves of modules.

\begin{definition}
    Let $X$ be a scheme with sheaf of rings $\cO_X$.
    A \textit{sheaf of $\cO_X$-modules} is a sheaf of modules $\cM$ such that for any $U \subseteq X$,
    $\cM(U)$ is an $\cO_X(U)$-module.
    Moreover, for an inclusion $V \subseteq U$ of open sets,
    for any $a \in \cO_X(U)$ and $m_1, m_2 \in \cM(U)$,
    we have
    \[
        (am_1 + m_2)|_V
        = a|_v m_1|_v + m_2|_v.
    \]
    The morphisms of such sheaves also preserve the structure as a $\cO_X$-module.
\end{definition}

\begin{example}
    Consider a closed immersion $\iota : Y \hookrightarrow X$.
    We may consider the \textit{ideal sheaf} $\cI_Y$ associated to $Y$ by
    \[
        \cI_Y = \ker(\cO_X \to \iota_* \cO_Y).
    \]
    Clearly, $\cI_Y(U) = \ker(\cO_X(U) \to \iota_* \cO_Y)$ is an ideal of $\cO_X(U)$, 
    and thus an $\cO_X(U)$-module, for any $U \subseteq X$ open.
    Moreover, the restriction map preserves the module structure since it's just a ring homomorphism on $\cO_X$ elements. 
\end{example}

Just as we may associate a structure sheaf $\cO_X$ to a ring $A$,
we may also associate a sheaf of $\cO_X$-modules to an $A$-module $M$.
In fact, one may do so in the following obvious way.

\begin{proposition}
    Fix a ring $A$ and an $A$-module $M$.
    If $(X, \cO_X) = \Spec A$,
    there exists an $\cO_X$-module $\wt M$ on $X$ associated $M$ which satisfies the following properties.
    \begin{enumerate}
        \item $\Gamma(X, \wt M) = M$.
        \item Given $f \in A$, $\Gamma(D_f, \wt M) \cong M_f$.
        \item For any $P \in X$, $\wt M_P \cong M_P$.
    \end{enumerate}
\end{proposition}
\begin{proof}
    See Proposition II.5.1 of \cite{Hartshorne_2013}.
\end{proof}

Although the affine case is quite useful,
we are more interested in how it applies to projective cases.
In order to do so, however,
we must first discuss graded modules.

\begin{definition}
    Let $S = \bigoplus_{d \in \Z} S_d$ be a graded ring.
    A \textit{graded $S$-module} $M = \bigoplus_{d \in \Z} M_d$ is such that $M_d$ is an $S_0$-module 
    and $S_e \cdot M_d \subseteq M_{d + e}$ for any $e, d \in \Z$.
    We also define the \textit{twisted module} $M(n)$ for each $n \in \Z$ as $M(n)_d = M_{n + d}$ for any $d \in \Z$.
\end{definition}

With this, we may describe how the affine case generalizes to our projective case.

\begin{proposition}
    Let $S$ be a graded ring and $M$ a graded $S$-module.
    Writing $(X, \cO_X) = \Proj(S)$,
    there exists a sheaf of $\cO_X$-modules $\wt M$
    which has the following characteristics.
    \begin{enumerate}
        \item For any $f \in S_+$,
        we have $\wt M|_{D_+(f)} \cong \wt{M_{(f)}}$,
        where $M_{(f)}$ is the submodule of $M_f$ that contains only degree zero elements.
        \item For any $P \in X$, $(\wt M)_P \cong M_{(p)}$,
        where the module $M_{(p)}$ is again the submodule of $M_p$ of elements of degree zero. 
    \end{enumerate} 
\end{proposition}
\begin{proof}
    See Proposition II.5.11 of \cite{Hartshorne_2013}.
\end{proof}

\begin{example}
    As a perfect example, let $S = k[x_0, \ldots, x_n]$ and $X = \Proj(S) = \bbP^n$.
    For each $n \geq 1$, we write $\cO_X(n)$ to denote the sheaf of $\cO_X$-modules associated to the graded module $\wt{S(n)}$.
    We also specifically refer to $\cO_X(1)$ as the \textit{twisting sheaf of Serre}.
    
    To understand the sheaf $\cO_X(1)$,
    we can look locally on each piece $D_+(x_i)$ for each $0 \leq i \leq n$.
    Specifically, note that
    \[
        S(n)_{(x_i)}
        = \bigcup_{d \geq 0}\{
            x_i^{-d}f(x_0, \ldots, x_n) : \deg f(x) = n + d
        \}
        = x_i^n k[\tfrac{x_0}{x_i}, \ldots, \tfrac{x_n}{x_i}].
    \]
    Thus, we see that $\cO_X(n)|_{D_+(x_i)} = x_i^n \cO_X$.
    Moreover, the global sections have the form
    \[
        \Gamma(X, \cO_X(n)) = \Span_k\{
            x_0^{a_0} \cdots x_n^{a_n} : 
            a_0 + \cdots + a_n = d
        \}.
    \]
\end{example}

From the previous example, 
which may be generalized to other graded rings $S$ generated by $S_1$ as an $S_0$ algebra (see Proposition II.5.12 of \cite{Hartshorne_2013}),
we note a few properties that make this sheaf nice.
To start, we note that our open cover revealed $\cO_X(n)$ to be \textit{free} on the subsets $U_i = D_+(x_i)$,
in the sense that it is isomorphic to $\bigoplus_{i = 1}^m \cO_x|_{U_i}$ with $m = 1$.
We refer to the exact integer on which our module is the direct sum copies of as the \textit{rank},
which must be constant on connected components.
When the rank is one,
we refer to our sheaf of modules as an \textit{invertible sheaf} for reasons which will become apparent.

Additionally, we refer to $\cO_X(n)$ as \textit{coherent},
which means that we may cover $X$ with open affine patches upon which $\cO_X|_U$ is the sheaf associated to a finitely generated module.
More generally, we may say that a sheaf of modules is \textit{quasi-coherent} when this condition holds without the module necessarily being finitely generated.

Examining the invertible sheaves further,
suppose that $\cM$ is an invertible sheaf on an integral scheme $X$.
In this case, let $\cup_{i \in I} U_i$ be an open cover of $X$ for which $\cM|_{U_i} \cong \cO_X|_{U_i}$ for all $i \in I$.
We notice that for any $i \in I$,
we may take some $s_i \in \Gamma(U_i, \cM)$ such that $\cM|_{U_i} = s_i \cO_X|_{U_i}$ since $s_i$ cannot restrict to $0$  on any open subset of $U_i$ without changing the entire structure of $\cM|_{U_i}$.
With this in mind, let's consider a pair $i, j \in I$ and the intersection $U_i \cap U_j$.
On $\Gamma(U_i \cap U_j)$, we find that both $s_i$ and $s_j$ generate the module,
and so there is some $f_{i,j}, f_{j, i} \in \cO_X(U_i \cap U_j)$ for which
\[
    s_i f_{i, j} = s_j, \qquad
    s_j f_{j, i} = s_i.
\]
In fact, we must have that $f_{i, j} = f_{j, i}^{-1}$ in the ring $\cO_X(U_i \cap U_j)$.
Fixing some $i_0 \in I$, and denoting $f_{i_0, i} = f_i$,
we may identify $\cM$ with a sub-$\cO_X$-module $\cL$ of $K(X)$,
now referring to $K(X)$ as the constant sheaf as a $\cO_X$-module,
by $\cL|_{U_i} = f_i \cO_X|_{U_i}$.

Interestingly, this in-fact takes us back to our previous discussion of ideal sheaves associated to closed subschemes,
specifically in the case where $f_i \in \cO_X(U_i)$,
where identification is done through localizing at the generic point.
In this case, our sheaf of ideals is locally principal, but we can speak more generally to this.

As an illustrative example, let's take $\cO_X(1)$ with $X = \bbP^2$.
Following the same procedure, we have the following association of open subsets to $K(X)$ as
\[
    D_+(x) \mapsto 1 \qquad D_+(y) \mapsto \frac{x}{y},
\]
so that $\cO(1)|_{D_+(x)} \cong \cO_X$ and $\cO(1)|_{D_+(y)} \cong \frac{x}{y} \cO_X|_{D_+(y)}$ under this isomorphism.
More generally, however, we may multiply the functions $1, \frac{x}{y} \in K(X)$ by any other $g \in K(X)^*$,
and we would still obtain an isomorphism.
Looking at the associated ideal sheaf,
it is quite clear that we have described $(0 : 1) \in \bbP^n$ as a closed subscheme.
And if we do take the liberty of using the functions $g$ and $g\tfrac{x}{y}$ for $g = \tfrac{y}{x} \in K(X)^*$,
we may move from $(0 : 1)$ to the point $(1 : 0)$.

Without knowing yet whether a given ideal sheaf of $K(X)$ corresponds to a closed subscheme,
there may be two distinct ideas we are looking at.
Namely, closed subschemes of codimension one, sub-$\cO_X$-modules of $K(X)$, and invertible sheaves.

\begin{definition}
    Let $X$ be an integral separated scheme such that for every $x \in X$,
    if $\dim \cO_x = 1$, then $\dim_{\kappa(x)} \Fm / \Fm^2 = \dim \cO_x = 1$.
    We refer to a closed integral subscheme $Y$ of $X$ as a \textit{prime divisor}.
    A \textit{Weil divisor} is a formal sum $D = \sum_{i = 1}^n a_i Y_i$,
    with $a_i \in \Z$ and $Y_i$ a prime divisor for each $1 \leq i \leq n$.
    
    The support of a Weil divisor $D$ is the union of prime divisors $Y$ for which the associated coefficient in $D$ is non-zero.
    We say that a divisor $D$ is \textit{effective}, and write $D \geq 0$, if the integers may be taken to be non-negative,
    and we identify such divisors with their support.


    As we may add and subtract these formal sums,
    the Weil divisors form a group, which we will denote as $\Div(X)$.
\end{definition}

To make sense as to how the group of Weil divisors operates,
consider the fact that a prime divisor $Y$ of $X$ has a generic point $\eta \in Y$.
By the codimension of $Y$, we find $\dim \cO_{X, \eta} = 1$,
and so by our regularity proposition,
$\cO_{X, \eta}$ is principal and thus a discrete valuation ring.
Therefore, on the units of our field of fractions of $K(X)$,
we may define an order of vanishing at the prime divisor $Y$,
denoted $v_Y : K(X)^* \to \Z$.
For a prime divisor $Y$, $f \in K(X)^*$, and an integer $m > 0$,
we say that $f$ has a \textit{zero of order $m$ at $Y$} if $v_Y(f) = m$,
and that $f$ has a \textit{pole of order $m$ at $Y$} if $v_Y(f) = -m$. 

\begin{definition}
    Given $X$ as before and $f \in K(X)^*$,
    we may define the \textit{principal divisor} $\rmDiv(f) \in \Div(X)$ as
    \[
        \rmDiv(f) = \sum_{\codim(Y, X) = 1} v_Y(f) Y.
    \]
    If $D_1, D_2 \in \Div(X)$ are such that $D_1 - D_2 = \rmDiv(f)$ for some $f \in K(X)^*$,
    we say that $D_1$ and $D_2$ are \textit{linearly equivalent} and may write $D_1 \sim D_2$.
\end{definition}

Just as we had noticed before in our example derived from invertible sheaves,
we see that there is some notion of equivalence by the elements of $K(X)^*$.

For another example, let $k/\Q$ be a finite extension with ring of integers $R_k$.
Since $R_k$ is a Dedekind domain, it is clear that $X = \Spec(R_k)$ satisfies all the defining properties required to describe $\Div(X)$.
Indeed, our principal divisors are just elements of $k^*$ and our prime divisors are prime ideals $P \subseteq R_k$.
We recall from Corollary 3.9 of \cite{Neukirch_2013} that given some finitely generated sub-$R_k$-module of $k^*$,
we may associate a unique factorization as prime ideals,
and this corresponds to the valuation along our prime divisors as before.
As expected, when we consider $\Div(X)$ under linear equivalence,
we have described the class group of $R_k$ which describes how far $R_k$ is from being a principal ideal domain.

\begin{definition}
    Let $X$ be such that we may define $\Div(X)$ as above.
    Then we may define the \textit{class group of divisors on $X$},
    denoted $\Cl(X)$,
    as the quotient of $\Div(X)$ by the subgroup of principal divisors.
\end{definition}

To obtain a more geometric picture of our class group of divisors,
consider an open subset $U \subseteq X$, with $\Cl(X)$ well-defined.
In the case where $\codim(X \sm U, X) \geq 2$,
notice that the prime divisors of $U$ will be given exactly as the restriction of a prime divisor from $X$,
and we likewise haven't removed any prime divisors in this process,
up to at least what may be permuted by the principal divisors.
However, if $\codim(X \sm U, X) = 1$, we have the following proposition.

\begin{proposition}
    Let $X$ be such that $\Cl(X)$ is well-defined,
    and suppose that $U \subseteq X$ is open with complement $Y$.
    Then $\Cl(U)$ is well-defined,
    and the restriction of the prime divisors of $X$ to $U$ grants a map $D \mapsto D|_U$ which is surjective.
    If $\codim(Y, X) \geq 2$, we have $\Cl(X) \cong \Cl(U)$.
    However, if $\codim(Y, X) = 1$, we have the following exact sequence,
    \[
        \langle Y \rangle \to \Cl(X) \to \Cl(U) \to 0,
    \]
    where the first map is just the inclusion as a subgroup. 
\end{proposition}

\begin{proof}
    Refer to Proposition II.6.5 of \cite{Hartshorne_2013}.
\end{proof}

As another hint in the direction towards unification of these ideas,
let $X$ be such that the class group is well-defined and take any $D \in \Div(X)$.
For each $U \subseteq X$, we may define a subgroup of $K(X)^*/\cO_X(U)^*$ as
\[
    L(D)(U) = \{ f \in K(X)^*/\cO_X(U)^* : D|_U + \rmDiv(f|_U) \geq 0 \}
\]
where we note the quotient may be taken as units of $\cO_X(U)$ certainly have no poles or zeroes on $U$.
Notice that linear equivalence, whether globally or locally on just the open set of interest,
induces an isomorphism between these subgroups by simply performing the group action of multiplication by the principal divisor on $K(X)^*/\cO_X(U)^*$ 

Also, we see that $L(D)(U) \cup \{0\}$ is an $\cO_X(U)$-module.
For any prime divisor $Y$, $f_1, f_2 \in L(D)(U)$, and $c_1 \in \cO_X(U)$ non-zero,
\[
    v_Y(c_1 f_1 + f_2)
    \geq \min(v_Y(c_1 f_1), v_Y(f_2))
    = \min(v_Y(f_1) \cdot v_Y(f_1), v_Y(f_2))
    \geq \min(v_Y(f_1), v_Y(f_2)).
\]
If $\cO_X(U)$ were to contain a field, this would then be a vector space,
as is the case with varieties.

While this may appear to be a sheaf,
in general this is not.
However, this may be easily accounted for.

\begin{definition}
    Let $X$ be an integral scheme.
    We recall that $K(X)$ acts as a constant sheaf on $X$.
    Taking $\cO_X^*$ to be the sheaf defined by the group of multiplicative units of $\cO$,
    a \textit{Cartier divisor} $D$ is a global section of $\Gamma(X, K(X)^*/\cO_X^*)$,
    where the quotient sheaf is defined as the sheaf associated to the presheaf $U \mapsto K(X)^*/\cO_X(U)^*$.
    The group $\Gamma(X, K(X)^*/\cO_X^*)$ is denoted $\CaDiv(X)$ and referred to as the group of Cartier divisors.

    Moreover, if a section lies in the image $K(X)^* \to K(X)^*/\cO_X^*$,
    we say that the Cartier divisor is \textit{principal}.
    Using additive language, when the difference between two Cartier divisors is principal,
    we say that they are \textit{linearly equivalent}.
    We denote the quotient of $\CaDiv(X)$ by the principal Cartier divisors as $\CaCl(X)$.
\end{definition}

By our explanation of a sheaf associated to a presheaf,
a Cartier divisor $D \in \CaDiv(X)$ may be represented by a set of pairs
\[
    D = \{(U_i, f_i) : f_i \in K(X)/\cO_X(U_i)^*, i \in I \}
\]
for which $\bigcup_{i \in I} U_i = X$,
and $f_i f_j^{-1} \in \cO_X(U_i \cap U_j)^*$ for any $i, j \in I$.
The group action on such sets is simply done by pairwise multiplication on intersections.

Just as with the Weil divisors,
the Cartier divisors as well have a similarly notated and similarly behaving associated group.
Taking $D = \{(U_i, f_i)\}_{i \in I} \in \CaDiv(X)$ for an integral scheme $X$,
we define the \textit{sheaf associated to $D$}, notated $\cL(D)$,
to be the sub-$\cO_X$-module of $K(X)^*$ which is defined locally as
\[
    \cL(D)|_{U_i} = \frac{1}{f_i} \cO_X|_{U_i},
\]
which may be glued together since $f_i/f_j^{-1} \in \cO_X(U_i \cap U_j)^*$ for any $i, j \in I$.
Furthermore, notice that $\cL(D + \rmDiv(f)) \cong \cL(D)$ by multiplication by $f \in K(X)^*$,
so it this sheaf is well-defined on $\CaDiv(X)$. 

Notice for the case of
\[
    D_W = (1 : 0) \in \Div(\bbP^1), \qquad 
    D_C = \{(D_+(x), 1), (D_+(y), \tfrac{x}{y})\} \in \CaDiv(\bbP^1),
\]
we immediately find a correspondence between $L(D_W)(U) \cong \Gamma(U, \cL(D_C)) \cong \Gamma(U, \cO(1))$ for any open $U \subseteq \bbP^1$.

\begin{definition}
    The \textit{Picard group} of $X$, denoted $\Pic(X)$,
    is the group of invertible sheaves of $\cO_X$-modules up to isomorphism.
    The group operation is given by $\cL_1 \otimes_{\cO_X} \cL_2$ for $\cL_1, \cL_2 \in \Pic(X)$,
    where $\cL_1 \otimes_{\cO_X} \cL_2$ is the sheaf associated to the presheaf
    \[
        U \mapsto cL_1(U) \otimes_{\cO_X(U)} \cL_2(U).
    \]
\end{definition}

\begin{theorem}
    Suppose that $X$ is a non-singular, integral, separated, noetherian scheme.
    Then there are group isomorphisms between $\Cl(X)$, $\CaCl(X)$, and $\Pic(X)$.
\end{theorem}

With our concepts joined together,
we may see how these groups allow us to understand a given non-singular variety $X/k$.
Take $D \in \Div(X)$ arbitrarily and consider the global sections of $\Gamma(X, \cL(D))$.
In the projective case, as $\Gamma(X, \cO_X) = k$,
we may consider a basis
\[
    \Gamma(X, \cL(D)) = \Span_k \{f_0, \ldots, f_n\},
\]
which is finite by A.3.2.7 of \cite{Silverman_Hindry_2013}.
Next, define a map $\varphi : X \to \bbP^n$ by
\[
    \varphi(P) = (f_0(P): \cdots : f_n(P)).
\]
The first issue which might be encountered is that one of our maps has a pole.
We may be able to move a pole around by multiplying through by $g \in K(X)^*$,
and clearly this will not change the value of $\varphi$ away from $\supp(\rmDiv(g))$.
However, we may encounter the issue as well that we could have a \textit{base point} of $D$ on which all global sections vanish.
If this does not occur, and the resulting map is a closed immersion,
then we say that $D$ is \textit{very ample},
while $D$ being \textit{ample} refers to the property that some multiple $mD$ is very ample for $m \geq 1$.

For an alternative picture, consider a map $\varphi : X/k \to Y/k$ of non-singular varieties.
Fixing $D \in \Pic(Y)$, represented as $\{(U_i, f_i)\}_{i \in I}$,
we would like to define
\[
    \varphi^* D = \{(\varphi^{-1}(U_i), f_i \circ \varphi)\}_{i \in I},
\]
where $f_i \circ \varphi$ is understood by considering $\eta \in X$ as the generic point and the map
\[
    \kappa(\varphi(\eta)) \to \kappa(\eta) = K(X)^*.    
\]
However, $\kappa(\varphi(\eta))$ may be a proper subset of $K(Y)^*$ in the case where $\varphi(X)$ is not dense in $Y$.
Fortunately, if all defining functions belong to $\kappa(\varphi(\eta))$,
this map makes sense,
and this is the condition that $\varphi(X)$ is not contained in the support of $D$.
It is known that for varieties by Lemma A.2.2.5 of \cite{Silverman_Hindry_2013}, 
we may always determine a representative $D'$ of the class of $D$ for which the pullback is defined.
Alternatively, we have for the associated sheaves,
\[
    \varphi^* \cL(D) = \varphi^{-1} \cL(D) \otimes_{\cO_Y} \cO_X.
\]
This allows us to express the following.

\begin{definition}
    Let $X$ be a scheme and $\cM$ a sheaf of $\cO_X$-modules.
    We say that $\cM$ is \textit{generated by global sections} if there exists an indexed set $\{s_i \}_{i \in I}$ of global sections of $\Gamma(X, \cM)$ such that 
    \[
        \cM_x = \sum_{i \in I} s_i|_x \cO_{X, x},
    \]
    for any $x \in X$.
\end{definition}

\begin{theorem}
    Let $X/k$ be a non-singular variety.
    \begin{enumerate}
        \item If $\varphi : X \to \bbP^n_k$ is a morphism,
        then $\varphi^*\cO(1)$ is an invertible sheaf generated by the global sections $\{x_i \circ \varphi\}_{i = 0}^n$.
        \item For any $D \in \Pic(X)$, if $\cL(D)$ is generated by global sections $s_0, \ldots, s_n \in \Gamma(X, \cL(D))$,
        then the map
        \[
            \varphi(x) = (s_0(x) : \cdots : s_n(x))
        \]
        is a morphism such that $\varphi^* \cO(1) \cong \cL(D)$ with $x_i \circ \varphi = s_i$ for each $0 \leq i \leq n$.
    \end{enumerate}
\end{theorem}

\begin{proof}
    See Theorem II.7.1 of \cite{Hartshorne_2013}.
\end{proof}

