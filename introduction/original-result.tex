We begin this section by stating a special case of the Subspace Theorem,
first introduced by Schmidt and generalized by Schlickewei to account for $p$-adic norms.
While we will explore the $p$-adic norms later, 
it is enough for now to note that for any prime $p \in \Z$ we obtain a norm
$|x|_p$ for any $x \in \Q$ by uniquely writing $x = p^k \frac{a}{b}$ with $a, b, k \in \Z$ and $\gcd(a, p) = \gcd(b, p) = \gcd(a, b) = 1$ and taking
\[
    |x|_p = p^{-k}.
\]
It should also be clear by unique factorization that
\[
    |x| \cdot \prod_{p \text{ prime}} |x|_p = 1.
\]
We sometimes write $|x|_\infty = |x|$ as our usual absolute value and collect all of these absolute values together into a set of places $M_\Q$.
It should be noted that the real application to algebraic numbers would require us to define similar absolute values on algebraic extensions of $\Q$,
which we will explore later.
We now recount a simplification the Subspace Theorem from \cite{Schlickewei_1977} as stated in \cite{BCZ_2002}.


\begin{theorem}
    Let $S \subseteq M_\Q$ be a finite set of places, 
    with $\infty \in S$ as to include the usual absolute value,
    and fix some integer $n \geq 1$.
    For each $v \in S$, take $L_{1,v}, \ldots, L_{n, v} : \Q^n \to \Q$ to be linearly independent set of linear forms.
    Fix some $\delta > 0$ and $C > 0$, 
    and consider the inequality
    \[
        \prod_{v \in S}\prod_{i = 1}^n |L_{i,v}(x_1, \ldots, x_n)|_v < (\max_{1 \leq i \leq n} |x_i|)^{-\delta}.
    \]
    Then the integer solutions $(x_1, \ldots, x_n) \in \Z^n$ with $\gcd(x_1, \ldots, x_n) = 1$ to the above inequality lie in finitely many hyperplanes of $\Q^n$.
\end{theorem}

Let's now go over the proof of the main result from \cite{BCZ_2002}.

\noindent\textit{Step 1: Setup}

Let $a, b \in \Z$ be multiplicatively independent integers. 
For each $n \geq 1$, we may take $D_n = \gcd(a^n - 1, b^n - 1)$,
and write
\[
    d_n = \frac{a^n - 1}{D_n}, \qquad 
    c_n = \frac{b^n - 1}{D_n}.
\]
With this, we clearly find for each $n \geq 1$
\[
    \frac{b^n - 1}{a^n - 1}
    = \frac{c_n}{d_n}.
\]

Fixing some $\varepsilon > 0$,
suppose for a contradiction that there is some infinite exceptional set $E$ for which $n \in E$ implies that $D_n \geq a^{\epsilon n}$.
Equivalently, note that this implies that $d_n \leq a^{(1 - \varepsilon)n}$ for exceptional $n$,
a fact that will be exploited by Diophantine approximation since our denominators will simply be too small for what we will ask of them. 

To get this proof moving, 
we will need some extra variables to allow for more flexibility in our approach.
For each $n, j \geq 1$, we also write
\[
    c_{n,j} 
    = d_n \cdot \frac{b^{jn} - 1}{a^n - 1}
    = \frac{b^{jn} - 1}{D_n}
    = \frac{b^n - 1}{D_n} \sum_{i = 0}^j b^{in}
    = c_n \sum_{i = 0}^j b^{in}.
\]
For the sake of convenience, we will write $z_{n,j} = c_{n,j} / d_n$.

\noindent\textit{Step 2: An initial bound}

To lean us towards an approximation,
we consider the series representation for each $n \geq 1$ as
\[
    \frac{1}{a^n - 1}
    = \frac{1}{a^n} \cdot \frac{1}{1 - (1/a)^n}
    = \frac{1}{a^n} \sum_{m = 0}^\infty \frac{1}{a^{mn}}
    = \sum_{m = 1}^\infty \frac{1}{a^{mn}}
\]
We recall that for any $M \geq 1$,
we have that the difference between the full series and the truncated series is
\begin{align*}
    \frac{1}{a^n - 1} - \sum_{m = 1}^M \frac{1}{a^{mn}}
    & = \sum_{m > M} \frac{1}{a^{mn}} \\
    & = \frac{1}{a^{nM}} \sum_{m = 1}^\infty \frac{1}{a^{mn}} \\
    & = \frac{1}{a^{nM}(a^n - 1)} \\
    & \ll \frac{1}{a^{n(M + 1)}}.
\end{align*}
Multiplying this difference by $(b^{jn} - 1)$ for a given $j \geq 1$, 
we obtain an approximation for $z_{n,j}$ as
\[
    \left|
        \frac{b^{jn} - 1}{a^n - 1}
        - \sum_{m = 1}^M \frac{b^{jn} - 1}{a^{mn}}
    \right|
    = \left|
        z_{n,j}
        - \sum_{m = 1}^M \frac{b^{jn}}{a^{mn}}
        + \sum_{m = 1}^M \frac{1}{a^{m n}}
    \right|
    \ll \frac{b^{jn}}{a^{n(M + 1)}}.
\]

\noindent\textit{Step 3: Trying Schmidt's Subspace Theorem}

Moving towards the Subspace Theorem,
notice that the previous approximation is a bound on a linear form in the variables $z_{n,j}$, $b^{jn}/a^{mn}$, and $1/a^{\ell n}$.
Fixing a finite number $J$ to only consider finitely many $1 \leq j \leq J$,
consider the space $\Q^N$ for $N = J + M + JM$ and coordinates
\[
    \Q^N = \{
        (x_1, \ldots, x_N) = (
            z_1, \ldots, z_J, 
            u_1, \ldots, u_M,
            v_{1,1}, \ldots, v_{J,M}
        ) : 
        z_j, u_m, v_{j,m}, x_i \in \Q
    \}.
\]
For each $n \geq 1$, we consider the vector $\vec{x}_n \in \Z^N$ with coordinates
\[
    \vec{x}_n = d_n a^{Mn} (
        z_{n,1}, \ldots, z_{n,J},
        \tfrac{1}{a^n}, \ldots, \tfrac{1}{a^{Mn}},
        \tfrac{b^n}{a^n}, \ldots, \tfrac{b^{nJ}}{a^{Mn}}
    ),
\]
where precisely $z_j = z_{n,j}$, $u_m = 1/a^{mn}$, and $v_{j,m} = b^{jn}/a^{mn}$ for each $1 \leq j \leq J$ and $1 \leq m \leq M$.
Note that the multiple of $d_n$ clears the denominators of the first group of coordinates,
and the multiple of $a^{Mn}$ clears the denominators of the rest of the coordinates.

We may now also consider linear form for each $1 \leq j \leq J$
\[
    L_{j, \infty}
    = z_j 
        + \sum_{m = 1}^M u_M
        - \sum_{m = 1}^M v_{j,m},
\]
so that our bound form the previous step shows $|L_{j, \infty}(\vec{x}_n)|_\infty \ll d_n b^{jn} / a^{n}$ for any $n \geq 1$.
Again towards the Subspace Theorem,
it also follows that for any $n \geq 1$ we have
\[
    \prod_{1 \leq j \leq J}
        |L_{j, \infty}(\vec{x}_n)|_\infty
    \ll \frac{d_n^J b^{J^2n}}{a^{Jn}}.
\]

In order to apply our Subspace Theorem,
note that we should have at least as many linear forms as the dimension of our input space.
Just as in the case of Roth's Theorem as a special case,
we will let the linear form $L_{i,\infty}$ for $i > J$ be given by $L_{i, \infty} = x_i$, 
accounting for the magnitude of the terms approximating $z_{n,j}$.

Clearly, all of our linear forms are linearly independent.
But taking their product, we find for any $n \geq 1$ that
\begin{align*}
    \prod_{1 \leq i \leq N} |L_{j, \infty}(\vec{x}_n)|_\infty
    & \ll \frac{d_n b^{J^2n}}{a^{Jn}}
        \cdot \prod_{1 \leq m \leq M} |a^{n(M - m)}|
        \cdot \prod_{\substack{
            1 \leq m \leq M \\
            1 \leq j \leq J
        }} |b^{jn}a^{n(M - m)}| \\
    & \ll \frac{d_n b^{J^2n}}{a^{Jn}}
        \cdot a^{M^2n}
        \cdot b^{J^2Mn}a^{JM^2n}
    = d_n b^{J^2(M + 1)n}a^{J(M^2 - 1)n}.
\end{align*}
Unfortunately, adding these extra linear forms shows that our approximation is not exceptional with respect to how large our inputs are,
even once we apply our bound to $d_n$.
However, this is only one norm for which our bound is not exceptional,
but we will see that there are norms on which our approximation is very good.

\noindent\textit{Step 4: Applying Schlickewei's Subspace Theorem}

With the previous shortcoming's in mind,
let's consider the set of places $S \subseteq M_\Q$ containing the standard absolute value,
but also containing the $p$-adic absolute values for each $p | ab$.
For ease of notation, let's represent $S^0 \subseteq S$ as the set of places in $S$ corresponding to a $p$-adic norm.
For each $1 \leq i \leq N$ and $v \in S^0$,
we will take $L_{i, v} = x_i$, which is clearly a linearly independent set of linear forms.
Recall that exponents of $a$ and $b$ are very small under the $p$-adic norms for $p | ab$,
so this will allow us to balance out our product relative to $S$.

Already, we may find that multiplying these linear forms balances out our product somewhat.
Specifically, for $i > J$ and $n \geq 1$, 
we find that $L_{i, v}(\vec{x}_n) = d_n a^{mn}b^{jn}$ for some $ 1 \leq m < M$ and $0 \leq j \leq J$ and so
\begin{align*}
    \prod_{v \in S} |L_{i, v}(\vec{x}_n)|_v
    & = \prod_{v \in S} |d_n|_v \cdot |a^{mn}b^{jn}|_v \\
    & = \left(
            \prod_{v \in S} |d_n|_v
        \right) \cdot
        \left(
            \prod_{v \in S} |a^{mn}b^{jn}|_v
        \right) \\
    & = |d_n| \left(
        \prod_{p | ab} |d_n|_p
        \right) \cdot
        |a^{mn}b^{jn}|
        \left(
            \prod_{p | ab} |a^{mn}b^{jn}|_p
    \right) \\
    & \leq d_n.
\end{align*}

Accounting for our remaining linear forms, 
for each $1 \leq j \leq J$, $v \in S^0$, and $n \geq 1$,
we recall that $L_{j, v}(\vec{x}_n) = d_n z_{n, j} a^{Mn} = c_{n, j} a^{Mn}$,
and so
\[
    \prod_{v \in S^0} |L_{j, v}(\vec{x})|_v
    = \prod_{p | ab} |c_{n, j}|_p |a^{Mn}|_p
    \leq \prod_{p | ab} |a^{Mn}|_p
    = \frac{1}{a^{Mn}}.
\]

Putting each of these bounds of these products together, 
we obtain that our full product for $x \in E$,
where we may assume $d_n \leq a^{(1 - \varepsilon)n}$,
is given by
\begin{align*}
    \prod_{v \in S} \prod_{1 \leq i \leq N} |L_{i, v}(\vec{x}_n)|_v
    & = \prod_{1 \leq j \leq J} |L_{j, \infty}(\vec{x}_n)|_\infty
    \cdot \prod_{1 \leq j \leq J} \prod_{v \in S^0} |L_{j, v}(\vec{x}_n)|_v
    \cdot \prod_{J < i \leq N} \prod_{v \in S} |L_{i, v}(\vec{x}_n)|_v \\
    & \ll \frac{d_n^J b^{J^2n}}{a^{Jn}}
    \cdot \frac{1}{a^{JMn}}
    \cdot d_n^{N - J} \\
    & \ll \frac{d_n^Nb^{J^2n}}{a^{J(M + 1)n}} \\
    & \ll \frac{a^{N(1 - \varepsilon)n}b^{J^2n}}{a^{(N - M)n}} \\
    & \ll \left(
            b^{J^2} a^{M - \varepsilon N}
        \right)^n
\end{align*}

To keep simplifying, we are ready to fix choices of $M$ and $J$.
Ideally, we would like to choose $J$ so that $M - \varepsilon N < J - M$,
and then by choosing $M$ for which $a^M > 2b^{J^2}a^{J}$ gives
\[
    \left(
        b^{J^2} a^{M - \varepsilon N}
    \right)^n
    < \left(
        b^{J^2} a^{J - M}
    \right)^n
    < \frac{1}{2^n}.
\]
Fortunately, $J > \frac{2}{\varepsilon}$ will suffice since this gives
\[
    M - \varepsilon N
    < M  + J - \varepsilon MJ
    < M + J - 2M
    = J - M.
\]

To obtain a bound in the form of the Subspace Theorem,
as the coordinates of $\vec{x}_n$ are given as rational polynomials in $a^n$, $b^n$, and $d_n$ of degrees no more than $MJ + 1$,
and $d_n \leq a^n$ is true for $n \in E$,
there is some constant $C > 1$ for which $\max_{1 \leq i \leq N} |\vec{x}_n| < C^n$.
Thus, by choosing $0 < \delta < \log 2 / \log C$
we must have for all sufficiently large $n \in E$ that
\[
    \prod_{v \in S} \prod_{1 \leq i \leq N} |L_{i, v}(\vec{x}_n)|_v
    < (\max_{1 \leq i \leq N} |\vec{x}_n|)^{-\delta}.
\]


\noindent\textit{Step 4: Too many integers, not enough hyperplanes}

With the work of setting up the inequality from the Subspace Theorem completed,
we may now reap the benefits of its conclusion.
This means that the vectors $\vec{x}_n$ for $n \in E$ must lie in only finitely many hyperplanes of $\Q^N$.
In particular, since $E$ is infinite,
there exists some infinite subset $E'$ for which $\vec{x}_n$ all lie on some rational hyperplane $H$.

To derive a contradiction,
we may explicitly write out our hyperplane as the equation
\[
    \sum_{j = 1}^J \zeta_j z_j
    + \sum_{m = 1}^J \alpha_1 u_1 
    + \sum_{j = 1}^J \sum_{m = 1}^J \beta_{j, m} v_{j,m}
    = 0,
\]
with $\zeta_k, \alpha_m, \beta_{j, m} \in \Q$ for $1 \leq j \leq J$ and $1 \leq m \leq M$.
Writing $\beta_{0,m} = \alpha_m$ for each $1 \leq m \leq M$,
notice that we have for any $n \in E'$,
\[
    \sum_{j = 1}^J \zeta_j \frac{b^{jn} - 1}{a^n - 1}
    + \sum_{\substack{
        0 \leq j \leq J \\
        1 \leq m \leq M 
    }} \beta_{j, m} \frac{b^{jn}}{a^{mn}}
    = 0,
\]
which follows easily by the coordinates of $\vec{x}_n$ and dividing out by $d_n a^{Mn}$.
That is, the integer polynomial
\[
    F(x, y) = x^M \sum_{j = 1}^J \zeta_j (y^j - 1)
    + (x - 1) \sum_{\substack{
        0 \leq j \leq J \\
        0 \leq m < M 
    }} \beta_{j, M - m} x^my^j
\]
has infinitely many roots along $\{(a^n, b^n) : n \in E'\}$.
However, this must mean that $F(x, y) = 0$ since we'd otherwise find a finite set of coefficients $\alpha_{i,j} \in \Q$ for $1 \leq i, j \leq K$,
such that for infinitely many $n \in E'$,
\[
    \sum_{1 \leq i, j, \leq K} \alpha_{i, j} a^{i n} b^{j n} = 0.
\]
Specifically, taking $1 \leq i_0, j_0 \leq k$ for which $\alpha{i_0,j_0} \neq 0$ and $a^{i_0}b^{j_0} > a^i b^j$ for any $i_0 \neq i, j_0 \neq j$ and $\alpha_{i,j} \neq 0$,
which may be done since we are assuming $a^i \neq b^j$ for all $i, j \geq 1$,
we find
\[
    \lim_{n \to \infty} \frac{1}{(a^{i_0}b^{j_0})^n} \sum_{1 \leq i, j, \leq K} \alpha_{i, j} a^{i n} b^{j n}
    = \alpha_{i_0, j_0},
\]
contradicting the fact that we may always find some $n \in E'$ large enough for which the term is zero.

Using the fact that $F(x, y) = 0$, and $\gcd(x^M, x - 1) = 1$,
we must have 
\[
    (x - 1)| \sum_{j = 1}^J \zeta_j (y^j - 1),
\]
which is only possible if each coefficient is zero.
Consequently, this must imply that $\beta_{j, M - m} = 0$ for each $0 \leq j \leq J$ and $0 \leq m < M$ as $(x - 1) \neq 0$.

Finally, we have deduced a contradiction as our vectors $\vec{x}_n$ do not lie on the trivial hyperplane, granting us the desired result.