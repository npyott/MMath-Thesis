While there were many technicalities,
we find overall there is particular relationship between the geometry of our spaces and the arithmetic of their rational points.
Indeed, while \cite{BCZ_2002} does initially focus on  Diophantine approximation,
a reinterpretation of the result under the geometry of $\bbP^n$ provides a generalization and even conjectures a result when you expand the set of places of $M_\Q$ of interest.
In a similar vein, McKinnon's theorem (\cite{McKinnon_2003}) on products of elliptic curves was able to prove a similar result on elliptic curve groups.

When we then generalize to Vojta's conjecture,
provided computation of the generalized GCD is available,
we have a reliable way of creating such conjectures on the GCD of arithmetic sequences and the geometry which gives rise to them.
In these cases, we note that the sequences $(x_n)_{n \geq 1}$,
either given as $x_n = a^n - 1$ for some $a \in \Z$,
or $x_n = D_{nP}$ with $P$ a rational point of infinite order on an elliptic curve,
follow a pattern that if $n | m$, then $x_n | x_m$.
Such sequences are referred to as \textit{divisibility sequences},
and Silverman conjectures in \cite{Silverman_2004} that sequences which arise from group schemes in the way of the previous two,
should have that the GCD returns to small values infinitely often.
However, the question over the integers whether the inequality, stated as
\[
    \gcd(a^n - 1, b^n - 1) \leq C,
\]
holds true for infinitely many $n \geq 1$,
assuming $a$ and $b$ multiplicatively independent and $C \geq 1$ arbitrary,
is currently a conjecture and not much is known.

A natural improvement on many of these results will be to first convert them into effective arguments,
and in this way,
more could be concluded from computation alone.
However, the cases which are still conditional on unproven cases of Vojta's conjecture remain elusive.
If we look to the future with an optimistic lens,
we may hope that the relationship between geometry and arithmetic continues to act as both a compass for navigating which results should hold,
as well a tool which permits their conclusion.