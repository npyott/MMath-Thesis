We begin this section by recalling our projective height definition on $\bbP^1$.
Fixing $(x : 1) \in \bbP(k) \sm \{ (0 : 1) \}$ for some finite extension $k / \Q$,
we recall
\[
    h(x : 1) 
    = h(1 : x^{-1}) 
    = \frac{1}{[k : \Q]}\sum_{v \in M_k} \max\{-\log \|x\|_v, 0\}.
\]
To simplify things,
we may consider for each $v \in M_k$ an associated function $v^+ : k_v^* \to \R_{\geq 0}$ given by ${v^+(x) = \frac{1}{[k : \Q]}\max\{-\log\|x\|_v, 0\}}$.
From here, it is clear that $v^+$ is morally behaving as minus the logarithm of $v$-adic distance function from $x \in k_v^*$ to $0 \in k_v$.

In particular, we see that the height $h(x : 1)$ is broken down as the $v$-adic distances from $(x: 1)$ to the point $(0 : 1)$.
Fundamentally, this is an interesting example since the divisor given by $(0 : 1)$ corresponds to the invertible sheaf $\cO(1)$,
which is exactly the identity embedding on $\bbP^1$ as we would hope.
Overall, we would like to show that we could think of our height functions as the sum of $v$-adic distances to the divisor in question.

To explore local heights further, fix a base field $k$ which is a finite extension of $\Q$,
and let $X/k$ be a projective variety.
For each $v \in M_k$, writing $k_v$ as the completion of $k$ with respect to this place and $v : k^\times \to \R$ as the function $v(x) = -\frac{1}{[k_v : \Q_v]} \log \|x\|_v$,
we may endow $X(k_v)$ with a minimal topology containing the Zariski open sets, 
and such that $P \mapsto \log v(f(P))$ is continuous away from the poles and zeroes of $f \in K(x)$. 

With this topology, for a given divisor $D$,
we would like to describe a family of functions $(\lambda_v)_{v \in M_k}$ such that
\[
    \lambda_v : (X \sm \supp(D))(k_v) \to \R
\]
is continuous and roughly approximates the negative logarithm of $v$-adic distance to $D$ in the sense that if $D|_U$ is given by $\rmDiv(f)$, we have

\[
    \sum_{v \in M_k} |\lambda_v(P) - v(f(P))| = O(1).
\]

\begin{definition}
    Let $X/k$ be a projective variety with $k$ a number field.
    A family of functions $(\gamma_v)_{v \in M_k}$, with $\gamma_v : X(k_v) \to \R$,
    is an \textit{$M_k$-constant} when each $\gamma_v$ is constant and $\gamma_v \neq 0$ for only finitely many $v \in M_k$.
    If we are given a similar family of functions $(\alpha_v)_{v \in M_K}$ and $(\beta_v)_{v \in M_k}$,
    we write $\alpha_v = \beta_v + O_v(1)$ if their difference is an $M_k$-constant.
\end{definition}

We also speak towards bounded regions.

\begin{definition}
    Let $X/k$ be a variety and consider a family of subset $S = (S_v)_{v \in M_k}$ for which $S_v \subseteq X(k_v)$.
    We say that the family $S$ is \textit{affine $M_k$-bounded} when there exists some affine open subset $U \subseteq X$ and $M_k$ constant $\gamma$,
    with coordinates $(x_1, \ldots, x_n)$,
    such that for any $v \in M_k$ and $(x_1, \ldots, x_n) \in S_v \cap U(k_v)$,
    \[
        \min_{1 \leq i \leq n}  v(x_i) \geq -\gamma_v.
    \]
    We say that a family $S$ is \textit{$M_k$-bounded} when we may cover our family with finitely many affine $M_k$-bounded families.
\end{definition}

\begin{proposition}
    Let $X/k$ be a projective variety. 
    Then the family $(X(k_v))_{v \in M_k}$ is $M_k$-bounded.
\end{proposition}

\begin{proof}
    See Proposition 10.1.2 of \cite{Lang_2013}.
\end{proof}

With the above in mind,
we would like to be able to write for a Cartier divisor $D = \{(U_i, f_i)\}$,
and associated family $\lambda = (\lambda_v)_{v \in M_k}$ for which we have for all $P \in U_i(k_v)$ which is neither a pole nor zero of $f_i$,
\[
    \lambda_v(P) - v(f(P))
\]
is continuous and bounded above and below by a predetermined $M_k$ constant on all $M_k$-bounded families of $U_i$.
This condition simplifies in projective space as we just require that we are bounded by an $M_k$-constant.
We often attribute a family of functions $\alpha_i = (\alpha_{i, v})_{v \in M_k}$ to represent this difference
\[
    \lambda_v = v \circ f_i + \alpha_{i,v},
\]
and would then require that $v \circ f_i f_j^{-1} = \alpha_{j, v} - \alpha_{i, v}$ on the intersections $U_i \cap U_j$.
Such functions are referred to as \textit{Weil functions}, or \textit{local height functions},
and we consequently obtain the following properties.

\begin{theorem}
    Let $X/k$ be a non-singular projective variety and $D$ a Cartier divisor.
    A Weil function $\lambda_D$ associated to $D$ satisfies the following properties.
    \begin{enumerate}
        \item (Uniqueness)
        If $\lambda'_D$ satisfies the same difference requirement as above,
        then $\lambda_{D,v} = \lambda'_{D,v} + O_v(1)$.

        \item (Additivity)
        For another divisor $E \in \Div(X)$, we have $\lambda_{D_1 + D_2, v} = \lambda_{D_1, v} + \lambda_{D_2, v} + O_v(1)$.

        \item (Functiorality)
        Given a morphism $\varphi: X \to Y$ of non-singular projective varieties and $E \in \Div(Y)$,
        we find that $\lambda_{\varphi^* E, v} = \lambda_{E, v} \circ \varphi + O_v(1)$
    \end{enumerate}
\end{theorem}

\begin{proof}
    The above properties are given in Proposition 10.2.1, Proposition 10.2.2, and Proposition 10.2.5 of \cite{Lang_2013}.
\end{proof}

To explore this relationship further, consider the following proposition.

\begin{proposition}
    Let $X/k$ be a non-singular projective variety,
    and let $D, D_1, \ldots, D_n \in \Div(X)$ a family of divisors such that $\cap_{1 \leq i \leq n} \supp(D_i) = \emptyset$.
    Then we may calculate for any $v \in M_k$ and $P \in (X \sm \supp(D))(k_v)$ that
    \[
        \lambda_{D, v} = \min_{1 \leq i \leq n} \lambda_{D + D_i, v}(P),
    \]
    where the minima only takes into account $1 \leq i \leq n$ for which $P \notin \supp(D_i)$. 
\end{proposition}

\begin{proof}
    See Proposition 10.3.2 in \cite{Lang_2013}.
\end{proof}

As a consequence, we may construct a Weil function for any divisor.
To do so, fix some $D \in \Div(X)$ and choose $E_1, \ldots, E_n$ and $F_1, \ldots, F_m \in \Div(X)$ such that 
\[
    \bigcap_{1 \leq i \leq n} \supp(E_i)
    = \bigcap_{1 \leq j \leq m} \supp(F_i)
    = \emptyset,
\]
and $D + E_i = F_j$ by Lemma 10.3.4 of \cite{Lang_2013}.
Next, taking $f_{i, j} \in K(X)$ such that $\rmDiv(f_{i,j}) = F_j - D - E_i$ for each $1 \leq i \leq n$ and $1 \leq j \leq m$,
we may use the following proposition to determine that for all $v \in M_k$, $1 \leq i \leq n$,  and $P \in (X \sm \supp(D + E_i))(k)$,
\[
    \lambda_{-D - E_i, v}(P)
    = \min_{1 \leq j \leq m} \lambda_{F_j - D - E_i, v}(P)
    = \min_{1 \leq j \leq m} v(f_{i,j}(P)),
\]
or $\lambda_{D + E_i} = \max_{1 \leq j \leq m} -v(f_{i,j}(P))$. 
With one more application of our proposition, we arrive at
\[
    \lambda_{D, v}(P)
    = \max_{1 \leq j \leq m} \min_{1 \leq i \leq n} -v(f_{i,j}(P)).
\]

This is indeed a very useful way to compute local heights.
In particular, if we have a hypersurface of $\bbP^n$ defined by a single equation $F(x_0, \ldots, x_n) = 0$ of degree $d$,
note that we have for each $0 \leq i \leq n$ that
\[
    d\{ x_i = 0\} - \{ F(x) = 0 \} = \rmDiv(x_i^d/F(x)).
\]
Since $\cap_{0 \leq i \leq n} \{ x_i = 0\} = \emptyset$, we then have
\[
    \lambda_{\{F(x) = 0\}}(x) = \max_{0 \leq j \leq m} v(F(x)/x_i^d).
\]

This very nice computational aspect aside,
we also have a few other properties which allow local height functions to apply quite generally.
Referring to Theorem B.8.1 of \cite{Silverman_Hindry_2013}, we write the following.

\begin{theorem}
    Consider a non-singular projective variety $X/k$ with divisor $D$.
    Let $D \mapsto \lambda_D$ be the local height machine as described above.
    We also have the following two properties.
    \begin{enumerate}
        \item (Positivity) If $D$ is effective, then $\lambda_{D,v} \geq O_v(1)$.
        \item (Local to Global Property) We may determine the Weil height $h_D$ by the local heights as
        \[
            h_D(P) = \sum_{v \in M_k} \frac{[k_v : \Q_v]}{[k : \Q]} \lambda_{D, v}(P) + O(1).
        \]
    \end{enumerate}
\end{theorem}