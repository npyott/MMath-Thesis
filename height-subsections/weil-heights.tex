With a notion for the arithmetic complexity of a point in projective space established,
we are now ready to extend this definition considerably. 
Indeed, there exists a very natural method to extend to a much wider class of geometries.

\begin{definition}
    Let $\varphi : X/k \to \bbP^n$ be a morphism.
    We may take the \textit{(absolute logarithmic) height on $X$ relative to $\varphi$} as $h : V(\ol{\Q}) \to [0, \infty)$ defined by $h_\varphi(P) = h(\varphi(P))$.
\end{definition}

One should note that for a given quasi-projective variety over $\ol{\Q}$,
there are always many maps which one can construct to projective space.
This does make sense, however,
since arithmetic complexity will naturally depend on the choices we are using to analyze our space,
just as much as it matters where we fix the origin on the number line when setting up height functions for the abstract curve $\bbP^1(\Q)$ itself.
Fortunately, when our choices of maps are similar with respect to the underlying geometry,
there is a strong relationship between the resulting height functions.

Indeed, for a projective variety $X/k$,
when $\varphi : X \to \bbP^n$ and $\psi : X \to \bbP^m$ are morphisms for which $\cL_1 = \phi^*\cO_{\bbP^n}(1)$ and $\cL_2 = \psi^*\cO_{\bbP^m}(1)$ are isomorphic,
we in particular have $\Gamma(X, \cL_1) \cong \Gamma(X, \cL_2)$ as a $k = \Gamma(X, \cO_X)$ module.
Thus, we may express the generating global sections of one module as a linear combination of global sections from another.
Then regarding these sheaves of modules as a $\cO_X$-module of $K(X)$,
the associated maps to projective space may be composed with a $k$-linear map to obtain the other.
As a result, we may state the following theorem.

\begin{theorem}
    Let $X/k$ be a projective variety and $k$ a number field,
    and consider two morphisms $\varphi : X \to \bbP^n$ and $\psi : X \to \bbP^m$.
    Suppose as well that $\phi^*\cO_{\bbP^n}(1) \cong \psi^*\cO_{\bbP^m}(1)$.
    Then there exists some constant $C > 0$ such that for any $P \in X(k)$,
    \[
        |h_\varphi(P) - h_\psi(P)| \leq C.
    \]
\end{theorem}

\begin{proof}
    In addition to the discussion above,
   we refer to Theorem B.3.1 of \cite{Silverman_Hindry_2013}.
\end{proof}

Using this fact, let's consider the maps associated to the base point free divisors on some variety $X/k$.
Immediately, we note that the choice of basis for $L(D)$ does not matter up to an $O(1)$ constant by the previous proposition.
Also, when $D_1 \sim D_2$ for $D_1, D_2 \in \Div(X)$,
it is also immediate that a basis $\{f_0, \ldots, f_n\}$ of $L(D_1)$ grants a basis $\{f_0g, \ldots, f_2g\}$ of $L(D_2)$ provided $D_1 = D_2 + \rmDiv(g)$,
so these maps agree away from the support of $\rmDiv(g)$ since
\[
    \varphi_D(x) = (f_0(x) : \cdots : f_n(x)) = (f_0(x)g(x) : \cdots : f_n(x)g(x)).
\]
Thus, we have a well-defined notion of heights associated to base point divisors in our Picard group up to $O(1)$, 
which we denote as $h_D$ for any $D \in \Div(X)$.

To extend this a bit, note that for any base point free divisors $D_1, D_2 \in \Div(X)$,
if we have $L(D_1) = \Span_k\{f_i\}_{i = 0}^n$ and $L(D_2) = \Span_k\{g_j\}_{0 = 1}^m$,
then there is some subset $I \subseteq \{(i, j) : 0 \leq i \leq n, 0 \leq j \leq m\}$ such that $L(D_1 + D_2) = \Span_k\{f_ig_j\}_{(i, j) \in I}$.
Using this, and the fact that there is some linear map $A : k^{|I|} \to k^{nm + n + m}$ such that $A((f_ig_j)_{_{(i, j) \in I}}) = (f_ig_j)_{0 \leq i \leq n, 0 \leq j \leq m}$,
for any $P \in X(k)$,
\begin{align*}
    [k : \Q] h_{D_1 + D_2}(P)
    & = [k : \Q] h(\phi_{D_1 + D_2}(P))(P) \\
    & = h(A(\phi_{D_1 + D_2}(P))) + O(1) \\
    & = \sum_{v \in M_k} \max_{\substack{
        0 \leq i \leq n \\
        0 \leq j \leq m
    }} \log \|f_i(P) g_j(P)\|_v + O(1) \\
    & = \sum_{v \in M_k} \max_{0 \leq i \leq n} \log \|f_i(P)\|_v
        + \sum_{v \in M_k}  \max_{0 \leq j \leq m} \log \|g_j(P)\|_v
        + O(1) \\
    & = [k : \Q](h_{D_1}(P) + h_{D_2}(P)) + O(1),
\end{align*}
implying $h_{D_1 + D_2} = h_{D_1} + h_{D_2}$.
Therefore, since we find that for any $D \in \Div(X)$,
we may always find base point free divisors $D_1, D_2 \in \Div(X)$ for which $D = D_1 - D_2$,
we may extend our notion to all such divisors in this additive way,
which is well-defined up to $O(1)$.
This map is referred to as \textit{Weil's Height Machine},
and we will explore how the geometric and algebraic way the divisor class group behaves on our space correspondingly affects the arithmetic of rational points.

While we have explored additivity, linear equivalence, and some aspects of uniqueness,
there are a few other properties which make our height machine very useful.

\begin{theorem}
    Let the map $D \mapsto h_D$ be the association of height functions to divisors described above.
    Letting $X$ and $Y$ be a variety over a number field $k$ and $D_1, D_2 \in \Div(X)$,
    we have the following properties.
    \begin{enumerate}
        \item (Uniqueness)
        The properties described here force a unique choice of height functions, 
        as constructed above, up to $O(1)$ equivalence.
        It is in fact possible as well to give effective bounds with respect to the defining equations of varieties, divisors, and morphisms.

        \item (Normalization)
        Let $H \subseteq \bbP^n$ be a hyperplane.
        Then $h_H = h + O(1)$,
        where $h : \bbP^n(\ol \Q) \to [0, \infty)$ is the absolute logarithmic height.

        \item (Functiorality)
        Let $\varphi : X \to Y$ be a morphism of varieties and fix $D \in \Div(Y)$,
       then $h_{\varphi^* D} = h_D \circ \varphi + O(1)$.

       \item (Additivity)
       $h_{D_1 + D_2} = h_{D_1} + h_{D_2} + O(1)$.

       \item (Linear Equivalence)
       If $D_1$ is linearly equivalent to $D_2$,
       then $h_{D_1} = h_{D_2} + O(1)$.

       \item (Positivity)
       If $D_1$ is effective and $U$ is an open subset on which none of the global sections of $\cL(D)$ vanish
       then $h_{D_1}|_U \geq O(1)$.

       \item (Northcott Property)
       Suppose that $D_1$ is ample.
       Then for any finite extension $k_2/k$ and $M > 0$,
       there are only finitely many $P \in V(k')$ for which $h_D(P) \leq B$.
    \end{enumerate}
\end{theorem}

\begin{proof}
    We refer to the proof of Theorem B.3.2 of \cite{Silverman_Hindry_2013} for more details.
\end{proof}